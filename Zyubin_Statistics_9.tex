% \documentclass[12pt]{article}
\documentclass[12pt]{amsart}

\pagestyle{plain}
\usepackage[margin=2cm]{geometry} 

\usepackage{amsmath,amssymb,amsfonts,enumerate,latexsym,amsthm,textcomp,wasysym}
\usepackage{nicefrac, xfrac}
\usepackage{hyperref}
\usepackage{subfiles}
%\usepackage{tocloft}

%\usepackage{indentfirst}
\usepackage{cancel}
\usepackage{graphicx}
% \graphicspath{{pictures/}}
% \DeclareGraphicsExtensions{.pdf,.png,.jpg}
%\usepackage{russian}

% Colors
\usepackage[dvipsnames]{xcolor}
\definecolor{linkcolor}{HTML}{0000FF} % цвет ссылок
\definecolor{urlcolor}{HTML}{0000FF} % цвет гиперссылок
\definecolor{citecolor}{HTML}{0000FF} % цвет ссылки на статью
\hypersetup{pdfstartview=FitH, linkcolor=linkcolor, urlcolor=urlcolor, citecolor=citecolor, colorlinks=true}
% Пробелы, отступы и выделения
\definecolor{todocolor}{HTML}{FF4500} % цвет todo
\definecolor{defcolor}{HTML}{EE5D0F} % цвет определений
\newcommand{\TODO}[1]{\textcolor{todocolor}{НУЖНО #1}}
\renewcommand\labelenumi{\rm (\arabic{enumi})}
\renewcommand\theenumi{\rm (\arabic{enumi})}
\definecolor{completed}{HTML}{32CD32}
\definecolor{inprocessing}{HTML}{D19A0F}

% Pictures and diagrams
\usepackage[matrix, arrow, curve]{xy} 
\usepackage{tikz-cd}
\usepackage{tikz}
\usetikzlibrary{shapes.geometric}
\usepackage{makecell}

\tikzset{
	symbol/.style={
		draw=none,
		every to/.append style={
			edge node={node [sloped, allow upside down, auto=false]{$#1$}}}
	}
}

\usepackage[utf8]{inputenc}
\usepackage[russian]{babel}
\usepackage{verbatim}
\makeatletter
\def\@settitle{\begin{center}%
		\baselineskip14\p@\relax
		\bfseries
		\large \@title
	\end{center}%
}
\makeatother


%%%%%%%%%%%%%%%%%%%%%%%%%%%%%%%%%%%%%%%%%%%%%%%%%%%%%%%%%%%%
% % commands for making comments
%\usepackage[dvipsnames]{xcolor}
\newcommand{\YP}[1]{\footnote{\textcolor{red}{YP: #1}}}
\newcommand{\yp}[1]{\leavevmode{\color{red}{#1}}}
% {\textcolor{orange}{#1}} 
\usepackage[normalem]{ulem}
%%%%%%%%%%%%%%%%%%%%%%%%%%%%%%%%%%%%%%%%%%%%%%%%%%%%%%%%%%%%

% \textheight=270mm
% \textwidth=190mm
% \voffset=-40mm
% \hoffset=-35mm
% \pagestyle{empty}
% 
% \\SLoppy

\emergencystretch=5pt

\setcounter{tocdepth}{4}
\setcounter{secnumdepth}{4}

% Theorems

\newtheorem{theorem}{Теорема}
\newtheorem*{definition}{Определение}
\newtheorem{proposition}[theorem]{Предложение}
\newtheorem{lemma}[theorem]{Лемма}
\newtheorem{corollary}[theorem]{Следствие}
\newtheorem*{remark*}{Замечание}


\theoremstyle{definition}

% Environments

\newenvironment{problem}[2][Problem name]{\indent \textcolor{#2}{\textbf{#1}} \indent}{\indent}
\newenvironment{squarestatement}[1][Statement]{\indent \textbf{[#1]} \indent}
{$ \hfill \lhd $ \indent}

% New Commands

% Set definition
\newcommand{\defineset}[2]{\left\{
	\left.
	#1 \
	\right\vert
	#2
	\right\}}

\newcommand{\Alt}{\mathfrak{A}}
\newcommand{\Sym}{\mathfrak{S}}
\newcommand{\D}{\mathrm{D}}
\newcommand{\Q}{\mathrm{Q}}
\newcommand{\rC}{\mathrm{C}}
\newcommand{\T}{\mathrm{T}}
\newcommand{\rO}{\mathrm{O}}
\newcommand{\I}{\mathrm{I}}
\newcommand{\CC}{\mathbb{C}}
\newcommand{\RR}{\mathbb{R}}
\newcommand{\FF}{\mathbb{F}}
\newcommand{\EE}{\mathbb{E}}

\newcommand{\calA}{\mathcal{A}}
\newcommand{\calB}{\mathcal{B}}
\newcommand{\calE}{\mathcal{E}}
\newcommand{\calP}{\mathcal{P}}

\newcommand{\GL}{\operatorname{GL}}
\newcommand{\SL}{\operatorname{SL}}
\newcommand{\PGL}{\operatorname{PGL}}
\newcommand{\PSL}{\operatorname{PSL}}
\newcommand{\SU}{\operatorname{SU}}
\newcommand{\SO}{\operatorname{SO}}
\newcommand{\diag}{\operatorname{diag}}
\newcommand{\characteristic}{\operatorname{char}}
\newcommand{\kk}{\Bbbk}
\newcommand{\Gal}{\mathrm{Gal}}

\newcommand{\Hom}{\mathrm{Hom}}
\newcommand{\projective}[1]{\mathrm{P^1}(#1)}
\newcommand{\Id}{\mathrm{Id}}
\newcommand{\Image}{\mathrm{Im} \,}
\newcommand{\Aut}[1]{\mathrm{Aut}\left(#1\right)}
\newcommand{\pr}[1]{\mathrm{pr}_{#1}}
\newcommand{\Rad}[1]{\mathrm{Rad}\left(#1\right)}
\newcommand{\Ann}[1]{\mathrm{Ann}\left(#1\right)}
\newcommand{\op}[1]{#1^{\mathrm{op}}}
\newcommand{\End}[2]{\mathrm{End}_{#2}\left(#1\right)}
\newcommand{\Ab}{\mathrm{Ab}}

\newcommand{\pp}{\mathfrak{p}}
\newcommand{\qq}{\mathfrak{q}}



% Кусочное определение функции
\newcommand{\definefuntwo}[4]{
	\begin{cases}
		#1, & #2; \\
		#3, & #4.
	\end{cases}
}

\newcommand{\definefunthree}[6]{
	\begin{cases}
		#1, & #2; \\
		#3, & #4; \\
		#5, & #6.
	\end{cases}
}

\newcommand{\definefunfour}[8]{
	\begin{cases}
		#1, & #2; \\
		#3, & #4; \\
		#5, & #6; \\
		#7, & #8.
	\end{cases}
}

\newcommand{\prob}{\operatorname{P}}
\newcommand{\events}{\mathfrak{F}}
\newcommand{\expect}{\operatorname{E}}
\newcommand{\disp}{\operatorname{D}}
\newcommand{\cov}{\operatorname{Cov}}

\newcommand{\params}{\Theta}

\newcommand{\red}[1]{{\color{red} #1}}
\newcommand{\blue}[1]{{\color{blue} #1}}

\title{Решения задач по математической статистике}
\author{Константин Зюбин}

%\pagenumbering{arabic}

\begin{document}
	
	\maketitle
	
	%	\tableofcontents
	
	%\section{Дз 2}
	
	\begin{problem}[Задача 1 ]{inprocessing}
		
		Пусть нулевая гипотеза $ H_0 $ состоит в том, что $ X_1 \sim \mathrm{R}[0, 1]$,
		а альтернатива $ H_1 $ заключается в том, что $ X_1 $ имеем функцию распределения 
		$ F_{X_1}(x) = \sin(\tfrac{\pi x}{2})I(0 < x < 1) + I(x \geqslant 1) $.
		
		Построим наиболее мощный критерий и вычислим мощность.
		
		Теперь $ \tfrac{L_1(x_1)}{L_0(x_1)} = \tfrac{\pi}{2}\cos(\tfrac{\pi x_1}{2}) $
		и неравенство $ \tfrac{\pi}{2}\cos(\tfrac{\pi x_1}{2}) > c_\alpha $
		выполнено  тогда и только тогда,
		когда $ x_1 < \tfrac{2\arccos(\frac{2c_\alpha}{\pi})}{\pi} $.
		Поэтому
		$$ \alpha = \prob_{0}(\tfrac{L_1(x_1)}{L_0(x_1)} > c_\alpha)
		= \prob_0(x_1 < \tfrac{2\arccos(\frac{2c_\alpha}{\pi})}{\pi})
		= \tfrac{2\arccos(\frac{2c_\alpha}{\pi})}{\pi}. $$
		Отсюда $ c_\alpha = \tfrac{\pi}{2}\cos(\tfrac{\pi \alpha}{2}) $.
		Таким образом, если $ \tfrac{L_1(x_1)}{L_0(x_1)} > \tfrac{\pi}{2}\cos(\tfrac{\pi \alpha}{2}) $,
		то нулевая гипотеза отклоняется.
		
		Вычислим мощность критерия.
		$$ \prob_{1}(\tfrac{L_1(x_1)}{L_0(x_1)} > \tfrac{\pi}{2}\cos(\tfrac{\pi \alpha}{2}))
		= \prob_{1}(\tfrac{\pi}{2}\cos(\tfrac{\pi x_1}{2} > \tfrac{\pi}{2}\cos(\tfrac{\pi \alpha}{2}))
		= \prob_{1}(\tfrac{\pi x_1}{2} < \tfrac{\pi \alpha}{2})
		= \prob_{1}(x_1 < \alpha)
		= \sin(\tfrac{\pi \alpha}{2}). $$
		
	\end{problem}
	
	\begin{problem}[Задача 2 (почему возникает достаточная статистика)]{inprocessing}
		
	По критерию факторизации для достаточной статистики $ T $ и некоторых функций $ g(\theta, t) $ и $ h(x_1, \ldots, x_n) $ 
	имеется равенство $ L(\theta, x_1, \ldots, x_n) = g(\theta, t)h(x_1, \ldots, x_n) $.
	При решении задач используется критерий Неймана-Пирсона,
	в построении которого участвует функция 
	$$ \tfrac{L(s, X_1, \ldots, X_n)}{L(s_0, X_1, \ldots, X_n)}
	= \tfrac{g(s, T)h(X_1, \ldots, X_n)}{g(s_0, T)h(X_1, \ldots, X_n)}
	= \tfrac{g(s, T)}{g(s_0, T)}. $$
	Тогда неравенство $ \tfrac{g(s, T)}{g(s_0, T)} > c_\alpha $ равносильно некоторому неравенству для $ T $.
		
	\end{problem}
	
	
	\begin{problem}[Задача 2аб ($ \mathrm{Exp}(1), \mathrm{Exp}(\tfrac{1}{s}) $)]{inprocessing}
	
	
		Пусть в соответствии с нулевой гипотезой случайные величины $ X_1, \ldots, X_n $
		имеют распределение $ \mathrm{Exp}(1) $.
		Альтернативной является гипотеза о распределении $ \mathrm{Exp}(\tfrac{1}{s}) $.
		Если $ X_1, \ldots, X_n \sim \mathrm{Exp}(\tfrac{1}{s}) $,
		то $ \overline{X} \sim \mathrm{\Gamma}(n, \tfrac{s}{n}) $.
		Вычислим функции правдоподобия в предположении нулевой гипотезы и альтернативы.
		$$ L_0(x_1, \ldots, x_n) = \prod\limits_{i = 1}^{n} e^{-x_i}I(X_{(1)} > 0); $$
		$$ L_1(x_1, \ldots, x_n) = \prod\limits_{i = 1}^{n} s^{-1}e^{-s^{-1}x_i}I(X_{(1)} > 0). $$
		Тогда
		$$ \tfrac{L_1(x_1, \ldots, x_n)}{L_0(x_1, \ldots, x_n)}
		= s^{-n}e^{-(s^{-1} -1)\sum\limits_{i = 1}^{n} x_i}. $$
		Отсюда
		$$ \alpha = \prob_{0}(\tfrac{L_1(X_1, \ldots, X_n)}{L_0(X_1, \ldots, X_n)} > c_\alpha)
		= \prob_{0}(-(s^{-1} - 1)\sum\limits_{i = 1}^{n} X_i > \ln c_\alpha + n\ln s). $$
		
		Если $ s < 1 $, то
		$$ \alpha = \prob_{0}(\overline{X} < \tfrac{\ln c_\alpha - n\ln s}{1 - s^{-1}}). $$
		Поэтому $ \tfrac{\ln c_\alpha - n\ln s}{1 - s^{-1}} = \gamma_{n, \frac{1}{n}, \alpha} $,
		где $ \gamma_{n, \frac{1}{n}, \alpha} $ --- квантиль уровня $ \alpha $
		для $ \mathrm{\Gamma}(n, \tfrac{1}{n}) $-распределения.
		Тогда $ c_\alpha = s^ne^{(1-s^{-1})\gamma_{n, \frac{1}{n}, \alpha}} $.
	
		Вычислим мощность критерия:
		$$ \beta = \prob_{1}(\tfrac{L_1(X_1, \ldots, X_n)}{L_0(X_1, \ldots, X_n)} > c_\alpha)
		= \prob_{1}(-(s^{-1} - 1)\sum\limits_{i = 1}^{n} X_i > \ln c_\alpha - n\ln s)
		= \prob_{1}(\overline{X} < \tfrac{\ln c_\alpha - n\ln s}{1 - s^{-1}}) = $$
		$$ = \prob_{1}(\overline{X} < \gamma_{n, \frac{1}{n}, \alpha})
		= \int\limits_{0}^{\gamma_{n, \frac{1}{n}, \alpha}} \tfrac{n^nx^{n - 1}}{\Gamma(n)}e^{nx}dx. $$
		
		Если $ s < 1 $, то
		$$ \alpha = \prob_{0}(\overline{X} > \tfrac{\ln c_\alpha - n\ln s}{1 - s^{-1}}). $$
		Поэтому $ \tfrac{\ln c_\alpha - n\ln s}{1 - s^{-1}} = \gamma_{n, \frac{1}{n}, 1 - \alpha} $,
		где $ \gamma_{n, \frac{1}{n}, 1- \alpha} $ --- квантиль уровня $ 1-\alpha $
		для $ \mathrm{\Gamma}(n, \tfrac{1}{n}) $-распределения.
		Тогда $ c_\alpha = s^ne^{(1-s^{-1})\gamma_{n, \frac{1}{n}, 1-\alpha}} $.
		
		Вычислим мощность критерия:
		$$ \beta = \prob_{1}(\tfrac{L_1(X_1, \ldots, X_n)}{L_0(X_1, \ldots, X_n)} > c_\alpha)
		= \prob_{1}(-(s^{-1} - 1)\sum\limits_{i = 1}^{n} X_i > \ln c_\alpha - n\ln s)
		= \prob_{1}(\overline{X} < \tfrac{\ln c_\alpha - n\ln s}{1 - s^{-1}}) = $$
		$$ = \prob_{1}(\overline{X} > \gamma_{n, \frac{1}{n}, \alpha})
		= 1 - \int\limits_{0}^{\gamma_{n, \frac{1}{n}, 1 - \alpha}} \tfrac{n^nx^{n - 1}}{\Gamma(n)}e^{nx}dx. $$
		
		Вычислим асимптотические квантили.
		Имеем $ \expect_0(\overline{X}) = 1 $, $ \disp_0(\overline{X}) = \tfrac{1}{n} $.
		Тогда $ \sqrt{n}(\overline{X} - 1) \overunderset{d}{n \to +\infty}{\to} \eta \sim \mathcal{N}(0, 1). $
		Поэтому 
		$$ \prob_0(\overline{X} < 1 + \tfrac{z_\alpha}{\sqrt{n}}) \to \alpha. $$
		
	\end{problem}
	
	\begin{problem}[Задача 2аб ($ \mathcal{N}(0, 1), \mathcal{N}(s, 1) $)]{inprocessing}
		
		Пусть в соответствии с нулевой гипотезой случайные величины $ X_1, \ldots, X_n $
		имеют распределение $ \mathcal{N}(0, 1) $.
		Альтернативной является гипотеза о распределении $ \mathcal{N}(s, 1) $.
		Если $ X_1, \ldots, X_n \sim \mathcal{N}(s, 1) $,
		то $ \overline{X} \sim \mathcal{N}(s, \tfrac{1}{n}) $.
		Вычислим функции правдоподобия в предположении нулевой гипотезы и альтернативы.
		$$ L_0(x_1, \ldots, x_n) = \prod\limits_{i = 1}^{n} \tfrac{1}{\sqrt{2\pi}}e^{-\tfrac{x_i^2}{2}}; $$
		$$ L_1(x_1, \ldots, x_n) = \prod\limits_{i = 1}^{n} \tfrac{1}{\sqrt{2\pi}}e^{-\tfrac{(x_i - s)^2}{2}}. $$
		Тогда
		$$ \tfrac{L_1(x_1, \ldots, x_n)}{L_0(x_1, \ldots, x_n)}
		= e^{\tfrac{2s\sum\limits_{i = 1}^{n} x_i - ns^2}{2}}. $$
		Отсюда
		$$ \beta = \prob_{0}(\tfrac{L_1(X_1, \ldots, X_n)}{L_0(X_1, \ldots, X_n)} > c_\alpha)
		= \prob_{0}(2s\overline{X} - s^2 > \tfrac{2\ln c_\alpha}{n})
		= \prob_{0}(2s\overline{X} > s^2 + \tfrac{2\ln c_\alpha}{n}). $$
		
		Если $ s > 0 $, то
		$$ \alpha = \prob_{0}(\overline{X} > \tfrac{s}{2} + \tfrac{\ln c_\alpha}{sn}). $$
		Поэтому $ \tfrac{s}{2} + \tfrac{\ln c_\alpha}{sn} = \tfrac{z_{1-\alpha}}{\sqrt{n}} $,
		где $ z_{\alpha} $ --- квантиль уровня $ \alpha $
		для $ \mathcal{N}(0, 1) $-распределения.
		Тогда $ c_\alpha = e^{\sqrt{n}sz_{1 - \alpha} - \frac{ns^2}{2}} $.
		
		Вычислим мощность критерия:
		$$ \beta(s) = \prob_{1}(\tfrac{L_1(X_1, \ldots, X_n)}{L_0(X_1, \ldots, X_n)} > c_\alpha)
		= \prob_{1}(2s\overline{X}-s^2 > \tfrac{2\ln c_\alpha}{n})
		= \prob_{1}(2s\overline{X} > s^2 + \tfrac{2\ln c_\alpha}{n}) = $$
		$$ = \prob_{1}(\overline{X} > \tfrac{s}{2} + \tfrac{\ln c_\alpha}{sn})
		= 1 - \Phi(\sqrt{n}(\tfrac{z_{1 - \alpha}}{\sqrt{n}} - s))
		= 1 - \Phi(z_{1 - \alpha} - \sqrt{n}s). $$
		
		Если $ s < 0 $, то
		$$ \alpha = \prob_{0}(\overline{X} < \tfrac{s}{2} + \tfrac{\ln c_\alpha}{sn}). $$
		Поэтому $ \tfrac{s}{2} + \tfrac{\ln c_\alpha}{sn} = \tfrac{z_{\alpha}}{\sqrt{n}} $,
		где $ z_{\alpha} $ --- квантиль уровня $ \alpha $
		для $ \mathcal{N}(0, 1) $-распределения.
		Тогда $ c_\alpha = e^{\sqrt{n}sz_{\alpha} - \frac{ns^2}{2}} $.
		
		Вычислим мощность критерия:
		$$ \beta = \prob_{1}(\tfrac{L_1(X_1, \ldots, X_n)}{L_0(X_1, \ldots, X_n)} > c_\alpha)
		= \prob_{1}(2s\overline{X}-s^2 > \tfrac{2\ln c_\alpha}{n})
		= \prob_{1}(2s\overline{X} > s^2 + \tfrac{2\ln c_\alpha}{n}) = $$
		$$ = \prob_{1}(\overline{X} < \tfrac{s}{2} + \tfrac{\ln c_\alpha}{sn})
		= \Phi(\sqrt{n}(\tfrac{z_{\alpha}}{\sqrt{n}} - s))
		= \Phi(z_{\alpha} - \sqrt{n}s). $$
		
		Вычислим асимптотические квантили.
		Имеем $ \expect_0(\overline{X}) = 0 $, $ \disp_0(\overline{X}) = \tfrac{1}{n} $.
		Тогда $ \sqrt{n}\overline{X} \overunderset{d}{n \to +\infty}{\to} \eta \sim \mathcal{N}(0, 1). $
		Поэтому 
		$$ \prob_0(\overline{X} < \tfrac{z_\alpha}{\sqrt{n}}) \to \alpha. $$
		
	\end{problem}
	
	
	\begin{problem}[Задача 2аб ($ \mathrm{Poiss}(1), \mathrm{Poiss}(s) $)]{inprocessing}
		
		Пусть в соответствии с нулевой гипотезой случайные величины $ X_1, \ldots, X_n $
		имеют распределение $ \mathrm{Poiss}(1) $.
		Если $ X_1, \ldots, X_n \sim \mathrm{Poiss}(s) $,
		то $ n\overline{X} \sim \mathrm{Poiss}(ns) $.
		Альтернативной является гипотеза о распределении $ \mathrm{Poiss}(s) $.
		Вычислим функции правдоподобия в предположении нулевой гипотезы и альтернативы.
		$$ L_0(x_1, \ldots, x_n) = \prob_\theta(X_1 = x_1, \ldots, X_n = x_n) 
		= e^{-n}\tfrac{1}{x_1!\ldots x_n!}; $$
		$$ L_1(s, x_1, \ldots, x_n)
		= e^{-ns}\tfrac{1}{x_1!\ldots x_n!}s^{(x_1 + x_2 + \ldots + x_n)}. $$
		Тогда
		$$ \tfrac{L_1(x_1, \ldots, x_n)}{L_0(x_1, \ldots, x_n)}
		= e^{n(1 - s)}s^{x_1 + x_2 + \ldots + x_n}. $$
		Отсюда
		$$ \alpha = \prob_{0}(\tfrac{L_1(X_1, \ldots, X_n)}{L_0(X_1, \ldots, X_n)} > c_\alpha)
		= \prob_{0}(n(1 - s) + n\overline{X}\ln s < c_\alpha)
		= \prob_{0}(n\overline{X}\ln s < c_\alpha + n(s - 1)). $$
		
		Если $ s > 1 $, то
		$$ \alpha = \prob_{0}(n\overline{X} < \tfrac{c_\alpha + n(s - 1)}{\ln s}). $$
		Поэтому $ \tfrac{c_\alpha + n(s - 1)}{\ln s} = p_{s, \alpha} $,
		где $ p_{n, \alpha} $ --- квантиль уровня $ \alpha $
		для распределения $ \mathrm{Poiss}(n) $.
		Тогда $ c_\alpha = p_{s, \alpha}\ln s - n(s - 1) $.
		
		Вычислим мощность критерия:
		$$ \beta = \prob_{1}(\tfrac{L_1(X_1, \ldots, X_n)}{L_0(X_1, \ldots, X_n)} > c_\alpha)
		= \prob_{1}(n(1 - s) + n\overline{X}\ln s < c_\alpha)
		= \prob_{1}(n\overline{X}\ln s < c_\alpha + n(s - 1)) = $$
		$$ = \prob_{1}(n\overline{X} < \tfrac{c_\alpha + n(s - 1)}{\ln s})
		= 1 - F_{ns}(p_{s, \alpha}), $$
		где $ F_{ns} $ --- функция распределения для случайной величины с распределением $ \mathrm{Poiss}(ns) $.
		
		Если $ s < 1 $, то
		$$ \alpha = \prob_{0}(n\overline{X} > \tfrac{c_\alpha + n(s - 1)}{\ln s}). $$
		Поэтому $ \tfrac{c_\alpha + n(s - 1)}{\ln s} = p_{n, 1-\alpha} $,
		где $ p_{n, \alpha} $ --- квантиль уровня $ \alpha $
		для распределения $ \mathrm{Poiss}(n) $.
		Тогда $ c_\alpha = p_{s, 1-\alpha}\ln s - n(s - 1) $.
		
		Вычислим мощность критерия:
		$$ \beta = \prob_{1}(\tfrac{L_1(X_1, \ldots, X_n)}{L_0(X_1, \ldots, X_n)} > c_\alpha)
		= \prob_{1}(n(1 - s) + n\overline{X}\ln s < c_\alpha)
		= \prob_{1}(n\overline{X}\ln s < c_\alpha + n(s - 1)) = $$
		$$ = \prob_{1}(n\overline{X} > \tfrac{c_\alpha + n(s - 1)}{\ln s})
		= 1 - F_{ns}(p_{s, 1-\alpha}), $$
		где $ F_{ns} $ --- функция распределения для случайной величины с распределением $ \mathrm{Poiss}(ns) $.
		
		Вычислим асимптотические квантили.
		Имеем $ \expect_0(n\overline{X}) = n $, $ \disp_0(n\overline{X}) = n $.
		Тогда $ \sqrt{n}(\overline{X} - n) \overunderset{d}{n \to +\infty}{\to} \eta \sim \mathcal{N}(0, 1). $
		Поэтому 
		$$ \prob_0(\overline{X} < n + \tfrac{z_\alpha}{\sqrt{n}}) \to \alpha. $$
		
	\end{problem}
	
	\begin{problem}[Задача 3аб]{inprocessing}
		
		Пусть $ X_1, \ldots, X_n \sim \mathcal{N}(\theta_1, \theta_2) $.
		
		Пусть нулевая гипотеза состоит в том, что $ \theta_1 = 0 $,
		а альтернатива заключается в том, что $ \theta_1 \neq 0 $.
		Построим критерий на основе $ g(\theta_1, X_1, \ldots, X_n) = \tfrac{\overline{X} - \theta_1}
		{\sqrt{\frac{1}{n - 1}\sum\limits_{i = 1}^{n} (X_i - \overline{X})^2}}. $
		Тогда при фиксированном $ \theta_1 $ случайная величина $ g\sqrt{n} $ имеет распределение $ t_{n - 1} $.
		Тогда
		$$ \prob_{0}(q_{n - 1, \frac{\alpha}{2}} < g\sqrt{n} < q_{n - 1, 1 - \frac{\alpha}{2}}) = \alpha $$
		и
		$$ \prob_{0}(\tfrac{q_{n - 1, \frac{\alpha}{2}}}{\sqrt{n}} < g < \tfrac{q_{n - 1, 1 - \frac{\alpha}{2}}}{\sqrt{n}}) = \alpha. $$
		Критерий уровня $ 1 - \alpha $ будет состоять в отклонении нулевой гипотезы при
		$ |g(0, X_1, \ldots, X_n)| > \tfrac{t_{n - 1, 1 - \frac{\alpha}{2}}}{\sqrt{n}} $.
		
		Пусть теперь нулевая гипотеза состоит в том, что $ \theta_2 = 1 $,
		а альтернатива утверждает обратное.
		Построим критерий на основе 
		$ h(X_1, \ldots, X_n) = \sum\limits_{i = 1}^{n} (X_i - \overline{X})^2. $
		Тогда при фиксированном $ \theta_2 $ случайная величина $ \tfrac{h}{(n - 1)\theta_2} $
		имеет распределение $ \chi^2_{n - 1} $.
		В предположении $ \theta_2 = 1 $ имеем
		$$ \prob_{0}(\tfrac{h}{n - 1} > c_{n - 1, \alpha}) = 1 - \alpha. $$
		Критерий уровня $ \alpha $ будет состоять в отклонении нулевой гипотезы при
		$ h(0, X_1, \ldots, X_n) > (n - 1)c_{n - 1, \alpha} $.	
		
	\end{problem}
	
\end{document}
