% \documentclass[12pt]{article}
\documentclass[12pt]{amsart}

\pagestyle{plain}
\usepackage[margin=2cm]{geometry} 

\usepackage{amsmath,amssymb,amsfonts,enumerate,latexsym,amsthm,textcomp,wasysym}
\usepackage{hyperref}
\usepackage{subfiles}
%\usepackage{tocloft}

%\usepackage{indentfirst}
\usepackage{cancel}
\usepackage{graphicx}
% \graphicspath{{pictures/}}
% \DeclareGraphicsExtensions{.pdf,.png,.jpg}
%\usepackage{russian}

% Colors
\usepackage[dvipsnames]{xcolor}
\definecolor{linkcolor}{HTML}{0000FF} % цвет ссылок
\definecolor{urlcolor}{HTML}{0000FF} % цвет гиперссылок
\definecolor{citecolor}{HTML}{0000FF} % цвет ссылки на статью
\hypersetup{pdfstartview=FitH, linkcolor=linkcolor, urlcolor=urlcolor, citecolor=citecolor, colorlinks=true}
% Пробелы, отступы и выделения
\definecolor{todocolor}{HTML}{FF4500} % цвет todo
\definecolor{defcolor}{HTML}{EE5D0F} % цвет определений
\newcommand{\TODO}[1]{\textcolor{todocolor}{НУЖНО #1}}
\renewcommand\labelenumi{\rm (\arabic{enumi})}
\renewcommand\theenumi{\rm (\arabic{enumi})}
\definecolor{completed}{HTML}{32CD32}
\definecolor{inprocessing}{HTML}{D19A0F}

% Pictures and diagrams
\usepackage[matrix, arrow, curve]{xy} 
\usepackage{tikz-cd}
\usepackage{tikz}
\usetikzlibrary{shapes.geometric}
\usepackage{makecell}

\tikzset{
	symbol/.style={
		draw=none,
		every to/.append style={
			edge node={node [sloped, allow upside down, auto=false]{$#1$}}}
	}
}

\usepackage[utf8]{inputenc}
\usepackage[russian]{babel}
\usepackage{verbatim}
\makeatletter
\def\@settitle{\begin{center}%
		\baselineskip14\p@\relax
		\bfseries
		\large \@title
	\end{center}%
}
\makeatother


%%%%%%%%%%%%%%%%%%%%%%%%%%%%%%%%%%%%%%%%%%%%%%%%%%%%%%%%%%%%
% % commands for making comments
%\usepackage[dvipsnames]{xcolor}
\newcommand{\YP}[1]{\footnote{\textcolor{red}{YP: #1}}}
\newcommand{\yp}[1]{\leavevmode{\color{red}{#1}}}
% {\textcolor{orange}{#1}} 
\usepackage[normalem]{ulem}
%%%%%%%%%%%%%%%%%%%%%%%%%%%%%%%%%%%%%%%%%%%%%%%%%%%%%%%%%%%%

% \textheight=270mm
% \textwidth=190mm
% \voffset=-40mm
% \hoffset=-35mm
% \pagestyle{empty}
% 
% \\SLoppy

\emergencystretch=5pt

\setcounter{tocdepth}{4}
\setcounter{secnumdepth}{4}

% Theorems

\newtheorem{theorem}{Теорема}
\newtheorem*{definition}{Определение}
\newtheorem{proposition}[theorem]{Предложение}
\newtheorem{lemma}[theorem]{Лемма}
\newtheorem{corollary}[theorem]{Следствие}
\newtheorem*{remark*}{Замечание}


\theoremstyle{definition}

% Environments

\newenvironment{problem}[2][Problem name]{\indent \textcolor{#2}{\textbf{#1}} \indent}{\indent}
\newenvironment{squarestatement}[1][Statement]{\indent \textbf{[#1]} \indent}
{$ \hfill \lhd $ \indent}

% New Commands

% Set definition
\newcommand{\defineset}[2]{\left\{
	\left.
	#1 \
	\right\vert
	#2
	\right\}}

\newcommand{\Alt}{\mathfrak{A}}
\newcommand{\Sym}{\mathfrak{S}}
\newcommand{\D}{\mathrm{D}}
\newcommand{\Q}{\mathrm{Q}}
\newcommand{\rC}{\mathrm{C}}
\newcommand{\T}{\mathrm{T}}
\newcommand{\rO}{\mathrm{O}}
\newcommand{\I}{\mathrm{I}}
\newcommand{\CC}{\mathbb{C}}
\newcommand{\RR}{\mathbb{R}}
\newcommand{\FF}{\mathbb{F}}
\newcommand{\EE}{\mathbb{E}}

\newcommand{\calA}{\mathcal{A}}
\newcommand{\calB}{\mathcal{B}}
\newcommand{\calE}{\mathcal{E}}
\newcommand{\calP}{\mathcal{P}}

\newcommand{\GL}{\operatorname{GL}}
\newcommand{\SL}{\operatorname{SL}}
\newcommand{\PGL}{\operatorname{PGL}}
\newcommand{\PSL}{\operatorname{PSL}}
\newcommand{\SU}{\operatorname{SU}}
\newcommand{\SO}{\operatorname{SO}}
\newcommand{\diag}{\operatorname{diag}}
\newcommand{\characteristic}{\operatorname{char}}
\newcommand{\kk}{\Bbbk}
\newcommand{\Gal}{\mathrm{Gal}}

\newcommand{\Hom}{\mathrm{Hom}}
\newcommand{\projective}[1]{\mathrm{P^1}(#1)}
\newcommand{\Id}{\mathrm{Id}}
\newcommand{\Image}{\mathrm{Im} \,}
\newcommand{\Aut}[1]{\mathrm{Aut}\left(#1\right)}
\newcommand{\pr}[1]{\mathrm{pr}_{#1}}
\newcommand{\Rad}[1]{\mathrm{Rad}\left(#1\right)}
\newcommand{\Ann}[1]{\mathrm{Ann}\left(#1\right)}
\newcommand{\op}[1]{#1^{\mathrm{op}}}
\newcommand{\End}[2]{\mathrm{End}_{#2}\left(#1\right)}
\newcommand{\Ab}{\mathrm{Ab}}

\newcommand{\pp}{\mathfrak{p}}
\newcommand{\qq}{\mathfrak{q}}



% Кусочное определение функции
\newcommand{\definefuntwo}[4]{
	\begin{cases}
		#1, & #2; \\
		#3, & #4.
	\end{cases}
}

\newcommand{\definefunthree}[6]{
	\begin{cases}
		#1, & #2; \\
		#3, & #4; \\
		#5, & #6.
	\end{cases}
}

\newcommand{\definefunfour}[8]{
	\begin{cases}
		#1, & #2; \\
		#3, & #4; \\
		#5, & #6; \\
		#7, & #8.
	\end{cases}
}

\newcommand{\prob}{\operatorname{P}}
\newcommand{\events}{\mathfrak{F}}
\newcommand{\expect}{\operatorname{E}}
\newcommand{\disp}{\operatorname{D}}
\newcommand{\cov}{\operatorname{Cov}}

\newcommand{\params}{\Theta}

\title{Решения задач по математической статистике}
\author{Константин Зюбин}

%\pagenumbering{arabic}

\begin{document}
	
	\maketitle
	
%	\tableofcontents
	
%\section{Дз 2}

\begin{problem}[Задача 1аб]{inprocessing}
	
	Пусть $ X_1, \ldots, X_n \sim \mathcal{R}[0, \theta] $ --- независимые случайные величины.
	Докажем, что оценки $ \sqrt[k]{(k + 1)\overline{X^k}} $ асимптотически нормальны для $ \theta $.
	
	Прежде проверим, что оценка (последовательность оценок) $ (k + 1)\overline{X^k} $ асимптотически нормальна для $ \theta $
	и вычислим её асимптотическую дисперсию.
	Согласно вычислениям в задаче 4 первой домашней работы $ \expect_\theta (k + 1)X_1^k = \theta^k $
	и $ \disp_\theta (k + 1)X_1^k = \tfrac{\theta^{2k} k^2}{2k + 1} $.
	Тогда по центральной предельной теореме
	$$ \sqrt{n}\left(\tfrac{1}{n}\sum\limits_{i = 1}^{n} (k + 1)X_i^k - \theta^k\right) 
	\overunderset{d}{n \to +\infty}{\to} \eta \sim \mathcal{N} (0, \tfrac{\theta^{2k} k^2}{2k + 1}). $$
	
	Пусть $ f(x) = \sqrt[k]{x} $. Тогда $ f'(x) = \tfrac{1}{k\sqrt[k]{x^{k - 1}}} $ не равно 0 при $ \theta > 0 $.
	По лемме об асимптотической нормальности для функции $ f(x) $
	получаем
	$$ \sqrt{n}\left(\sqrt[k]{\tfrac{1}{n}\sum\limits_{i = 1}^{n} (k + 1)X_i^k} - \theta\right) 
	\overunderset{d}{n \to +\infty}{\to} \tilde{\eta}
	\sim \mathcal{N} (0, \tfrac{\theta^{2k} k^2}{2k + 1} \cdot (\tfrac{1}{k\theta^{k - 1}})^2). $$
	Асимптотическая дисперсия равна 
	$$ \tfrac{\theta^{2k} k^2}{2k + 1} \cdot (\tfrac{1}{k\theta^{k - 1}})^2
	= \tfrac{\theta^2}{2k + 1}. $$
	Поскольку для $ m > k $ выполнено неравенство $ \tfrac{\theta^2}{2m + 1} < \tfrac{\theta^2}{2k + 1} $,
	то среди оценок такого вида нет оценки с наименьшей (при фиксированном $ \theta $) асимптотической дисперсией.
 	
\end{problem}

\begin{problem}[Задача 2а]{inprocessing}
	
	Пусть $ X_1, \ldots, X_n \sim \mathcal{R}[0, \theta] $ --- независимые случайные величины.
	Проверим, что оценка $ X_{(n)} = \max(X_1, \ldots, X_n) $ состоятельна для $ \theta $.
	Согласно вычислениям, выполненным ранее, имеем формулу
	$$ F_{X_{(n)}, \theta}(x) 
	= \definefunthree
	{0}{x < 0}
	{\tfrac{x^n}{\theta^n}}{x \in [0, \theta]}
	{1}{x > \theta} $$
	Поэтому для малых $ \varepsilon > 0 $ имеем 
	$$ \prob_\theta(|X_{(n)} - \theta| > \varepsilon)
	= 1 - F_{X_{(n)}}(\theta + \varepsilon) + F_{X_{(n)}}(\theta - \varepsilon)
	= 1 - 1 + \tfrac{(\theta - \varepsilon)^n}{\theta^n} = \left(1 - \tfrac{\varepsilon}{\theta}\right)^n. $$
	Последнее выражение стремится к 0 при $ n \to +\infty $, так как $ |1 - \tfrac{\varepsilon}{\theta}| < 1 $.
	Следовательно, последовательность $ X_{(n)} $ сходится по вероятности к $ \theta $
	и, по определению, состоятельна для $ \theta $.
	
\end{problem}

\begin{problem}[Задача 2б]{inprocessing}
	
	Найдём функцию распределения величины $ n(\theta - X_{(n)}) $:
	$$ F_{n(\theta - X_{(n)}), \theta}(x)
	= \prob_\theta(n(\theta - X_{(n)}) \leqslant x)
	= \prob_\theta(X_{(n)} \geqslant \theta - \tfrac{x}{n})
	= 1 - F_{X_{(n)}, \theta}(\theta - \tfrac{x}{n}) = $$
	$$ =
	\definefunthree
	{0}{x < 0}{1 - (1 - \tfrac{x}{n\theta})^n}{x \in [0, n\theta]}{1}{x > n\theta} $$
	
	При $ n \to +\infty $ данная последовательность функций распределений сходится к функции экспоненциального распределению
	$$ F(x) = \definefuntwo
	{0}{x < 0}{1 - e^{-\tfrac{x}{\theta}}}{x \geqslant 0} $$
	Таким образом, $ n(\theta - X_{(n)}) \overunderset{d}{n \to +\infty}{\to} \xi \sim \mathrm{E}(\tfrac{1}{\theta}) $.
	
\end{problem}

\begin{problem}[Задача 3а]{inprocessing}
	
	Пусть независимые случайные величины $ X_1, \ldots, X_n \sim \mathcal{N}(\theta, 1) $ имеют стандартное нормальное распределение.
	Случайная величина $ S_n = \sum\limits_{i = 1}^{n} X_i $, 
	как свёртка $ n $ независимых нормально распределённых случайных величин, имеет распределение $ \mathcal{N}(n\theta, n) $.
	Тогда случайная величина $ \overline{X} = \tfrac{1}{n}S_n $ имеет распределение $ \mathcal{N}(\theta, \tfrac{1}{n}) $.
	
\end{problem}

\begin{problem}[Задача 3б]{inprocessing}
	
	Докажем, что при $ \theta \neq 0 $ последовательность случайных величин $ \overline{X}^2 $ является асимптотически нормальной для $ \theta^2 $.
	
	По центральной предельной теореме имеем сходимость
	$$ \sqrt{n}(\overline{X} - \theta) \overunderset{d}{n \to +\infty}{\to} \eta \sim \mathcal{N}(0, 1). $$
	Положим $ f(x) = x^2 $. При $ \theta \neq 0 $ производная $ f'(x) = 2x $ не равна 0, поэтому по лемме об асимптотической нормальности имеем
	$$ \sqrt{n}(\overline{X}^2 - \theta^2) \overunderset{d}{n \to +\infty}{\to} \tilde{\eta} \sim \mathcal{N}(0, (2\theta)^2). $$
	
\end{problem}

\begin{problem}[Задача 3в]{inprocessing}
	
	В случае $ \theta = 0 $ рассмотрим функцию распределения случайной величины $ \sqrt{n}\overline{X}^2 $ при $ n \to +\infty $:
		
	$$ F_{\sqrt{n}\overline{X}^2}(x) 
	= \iint_{x_1x_2 \leqslant \tfrac{x}{\sqrt{n}}} \tfrac{n}{2\pi}e^{-\tfrac{n(x_1^2 + x_2^2)}{2}}dx_1dx_2
	= \iint_{u_1u_2 \leqslant \sqrt{n}x} \tfrac{1}{2\pi}e^{-\tfrac{u_1^2 + u_2^2}{2}}du_1du_2. $$
	Для $ x > 0 $ имеем 
	$$ \iint_{u_1u_2 \leqslant \sqrt{n}x, u_1^2 + u_2^2 \leqslant n} \tfrac{1}{2\pi}e^{-\tfrac{u_1^2 + u_2^2}{2}}du_1du_2
	\leqslant  \iint_{u_1u_2 \leqslant \sqrt{n}x} \tfrac{1}{2\pi}e^{-\tfrac{u_1^2 + u_2^2}{2}}du_1du_2 \leqslant 1. $$
	Поскольку интеграл сходится абсолютно на всей плоскости, то при $ x > 0 $ выражение в левой части неравенства при 
	$ n \to +\infty $ стремится к 1, поэтому при $ x > 0 $ выполнено 
	$$ \lim\limits_{n \to +\infty} F_{\sqrt{n}\overline{X}^2}(x) = 1. $$
	Пpи $ x < 0 $ имеем
	$$ 0
	\leqslant \iint_{u_1u_2 \leqslant \sqrt{n}x} \tfrac{1}{2\pi}e^{-\tfrac{u_1^2 + u_2^2}{2}}du_1du_2 
	= \iint_{-u_1u_2 \geqslant -\sqrt{n}x} \tfrac{1}{2\pi}e^{-\tfrac{u_1^2 + u_2^2}{2}}du_1du_2
	\leqslant \iint_{u_1^2 + u_2^2 \geqslant -\sqrt{n}x} \tfrac{1}{2\pi}e^{-\tfrac{u_1^2 + u_2^2}{2}}du_1du_2. $$
	Правая часть двойного неравенства стремится к 0 при $ n \to +\infty $, 
	поэтому	для $ x < 0 $ имеем
	$$ \lim\limits_{n \to +\infty} F_{\sqrt{n}\overline{X}^2}(x) = 0. $$
	Таким образом, функции распределения $ F_{\sqrt{n}\overline{X}^2} $ сходятся к функции распределения тождественно нулевой случайной величины и $ \sqrt{n}\overline{X}^2 \overunderset{d}{n \to +\infty}{\to} 0 $.	
		
\end{problem}

\begin{problem}[Задача 4а]{inprocessing}
	
	Пусть даны независимые случайные величины $ X_1, \ldots, X_n \sim \mathrm{Be}(\theta) $, где $ \theta \in (0, 1) $.
	Поскольку $ \expect X_i = \theta $ и $ \disp X_i = \theta(1 - \theta) $,
	то по центральной предельной теореме имеем
	$$ \sqrt{n}(\overline{X} - \theta) \overunderset{d}{n \to +\infty}{\to} \eta \sim \mathcal{N}(0, \theta(1 - \theta)). $$
	Положим $ f(x) = e^x $. Тогда производная $ f'(x) = e^x $ нигде не равна 0 и по лемме об асимптотической нормальности
	$$ \sqrt{n}(e^{\overline{X}} - e^\theta) \overunderset{d}{n \to +\infty}{\to} \tilde{\eta} \sim \mathcal{N}(0, \theta(1 - \theta) \cdot e^{2\theta}). $$
	Асимптотическая  дисперсия равна $ \theta(1 - \theta)e^{4\theta^2} $.
	
\end{problem}

\begin{problem}[Задача 4б]{inprocessing}
	
	Пусть даны независимые случайные величины $ X_1, \ldots, X_n \sim \mathrm{Poiss}(\theta) $, где $ \theta > 0 $.
	Поскольку $ \expect X_i = \theta $ и $ \disp X_i = \theta $,
	то по центральной предельной теореме имеем
	$$ \sqrt{n}(\overline{X} - \theta) \overunderset{d}{n \to +\infty}{\to} \eta \sim \mathcal{N}(0, \theta). $$
	Положим $ f(x) = x^3 $. Тогда производная $ f'(x) = 3x^2 $ не равна 0 при $ \theta > 0 $. 
	По лемме об асимптотической нормальности
	$$ \sqrt{n}({\overline{X}}^3 - \theta^3) \overunderset{d}{n \to +\infty}{\to} \tilde{\eta} \sim \mathcal{N}(0, \theta\cdot (3\theta^2)^2). $$
	Асимптотическая  дисперсия равна $ 9\theta^5 $.
	
\end{problem}

\begin{problem}[Задача 4в]{inprocessing}
	
	Пусть даны независимые случайные величины $ X_1, \ldots, X_n \sim \mathrm{Geom}(\theta) $, где $ \theta \in (0, 1) $.
	Поскольку $ \expect X_i = \tfrac{1}{\theta} $ и $ \disp X_i = \tfrac{1 - \theta}{\theta^2} $,
	то по центральной предельной теореме имеем
	$$ \sqrt{n}(\overline{X} - \tfrac{1}{\theta}) \overunderset{d}{n \to +\infty}{\to} \eta 
	\sim \mathcal{N}(0, \tfrac{1 - \theta}{\theta^2}). $$
	Положим $ f(x) = \tfrac{1}{x^2} $. Тогда производная $ f'(x) = \tfrac{-2}{x^3} $ не равна 0 при $ \theta \in (0, 1) $. 
	По лемме об асимптотической нормальности
	$$ \sqrt{n}({\overline{X}}^{-2} - \theta^2) \overunderset{d}{n \to +\infty}{\to} \tilde{\eta} \sim \mathcal{N}(0, \tfrac{1 - \theta}{\theta^2} \cdot (-2\theta^3)^2). $$
	Асимптотическая  дисперсия равна $ 4(1 - \theta)\theta^4 $.
	
\end{problem}

\end{document}
