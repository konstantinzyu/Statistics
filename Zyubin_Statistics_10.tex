% \documentclass[12pt]{article}
\documentclass[12pt]{amsart}

\pagestyle{plain}
\usepackage[margin=2cm]{geometry} 

\usepackage{amsmath,amssymb,amsfonts,enumerate,latexsym,amsthm,textcomp,wasysym}
\usepackage{nicefrac, xfrac}
\usepackage{hyperref}
\usepackage{subfiles}
%\usepackage{tocloft}

%\usepackage{indentfirst}
\usepackage{cancel}
\usepackage{graphicx}
% \graphicspath{{pictures/}}
% \DeclareGraphicsExtensions{.pdf,.png,.jpg}
%\usepackage{russian}

% Colors
\usepackage[dvipsnames]{xcolor}
\definecolor{linkcolor}{HTML}{0000FF} % цвет ссылок
\definecolor{urlcolor}{HTML}{0000FF} % цвет гиперссылок
\definecolor{citecolor}{HTML}{0000FF} % цвет ссылки на статью
\hypersetup{pdfstartview=FitH, linkcolor=linkcolor, urlcolor=urlcolor, citecolor=citecolor, colorlinks=true}
% Пробелы, отступы и выделения
\definecolor{todocolor}{HTML}{FF4500} % цвет todo
\definecolor{defcolor}{HTML}{EE5D0F} % цвет определений
\newcommand{\TODO}[1]{\textcolor{todocolor}{НУЖНО #1}}
\renewcommand\labelenumi{\rm (\arabic{enumi})}
\renewcommand\theenumi{\rm (\arabic{enumi})}
\definecolor{completed}{HTML}{32CD32}
\definecolor{inprocessing}{HTML}{D19A0F}

% Pictures and diagrams
\usepackage[matrix, arrow, curve]{xy} 
\usepackage{tikz-cd}
\usepackage{tikz}
\usetikzlibrary{shapes.geometric}
\usepackage{makecell}

\tikzset{
	symbol/.style={
		draw=none,
		every to/.append style={
			edge node={node [sloped, allow upside down, auto=false]{$#1$}}}
	}
}

\usepackage[utf8]{inputenc}
\usepackage[russian]{babel}
\usepackage{verbatim}
\makeatletter
\def\@settitle{\begin{center}%
		\baselineskip14\p@\relax
		\bfseries
		\large \@title
	\end{center}%
}
\makeatother


%%%%%%%%%%%%%%%%%%%%%%%%%%%%%%%%%%%%%%%%%%%%%%%%%%%%%%%%%%%%
% % commands for making comments
%\usepackage[dvipsnames]{xcolor}
\newcommand{\YP}[1]{\footnote{\textcolor{red}{YP: #1}}}
\newcommand{\yp}[1]{\leavevmode{\color{red}{#1}}}
% {\textcolor{orange}{#1}} 
\usepackage[normalem]{ulem}
%%%%%%%%%%%%%%%%%%%%%%%%%%%%%%%%%%%%%%%%%%%%%%%%%%%%%%%%%%%%

% \textheight=270mm
% \textwidth=190mm
% \voffset=-40mm
% \hoffset=-35mm
% \pagestyle{empty}
% 
% \\SLoppy

\emergencystretch=5pt

\setcounter{tocdepth}{4}
\setcounter{secnumdepth}{4}

% Theorems

\newtheorem{theorem}{Теорема}
\newtheorem*{definition}{Определение}
\newtheorem{proposition}[theorem]{Предложение}
\newtheorem{lemma}[theorem]{Лемма}
\newtheorem{corollary}[theorem]{Следствие}
\newtheorem*{remark*}{Замечание}


\theoremstyle{definition}

% Environments

\newenvironment{problem}[2][Problem name]{\indent \textcolor{#2}{\textbf{#1}} \indent}{\indent}
\newenvironment{squarestatement}[1][Statement]{\indent \textbf{[#1]} \indent}
{$ \hfill \lhd $ \indent}

% New Commands

% Set definition
\newcommand{\defineset}[2]{\left\{
	\left.
	#1 \
	\right\vert
	#2
	\right\}}

\newcommand{\Alt}{\mathfrak{A}}
\newcommand{\Sym}{\mathfrak{S}}
\newcommand{\D}{\mathrm{D}}
\newcommand{\Q}{\mathrm{Q}}
\newcommand{\rC}{\mathrm{C}}
\newcommand{\T}{\mathrm{T}}
\newcommand{\rO}{\mathrm{O}}
\newcommand{\I}{\mathrm{I}}
\newcommand{\CC}{\mathbb{C}}
\newcommand{\RR}{\mathbb{R}}
\newcommand{\FF}{\mathbb{F}}
\newcommand{\EE}{\mathbb{E}}

\newcommand{\calA}{\mathcal{A}}
\newcommand{\calB}{\mathcal{B}}
\newcommand{\calE}{\mathcal{E}}
\newcommand{\calP}{\mathcal{P}}

\newcommand{\GL}{\operatorname{GL}}
\newcommand{\SL}{\operatorname{SL}}
\newcommand{\PGL}{\operatorname{PGL}}
\newcommand{\PSL}{\operatorname{PSL}}
\newcommand{\SU}{\operatorname{SU}}
\newcommand{\SO}{\operatorname{SO}}
\newcommand{\diag}{\operatorname{diag}}
\newcommand{\characteristic}{\operatorname{char}}
\newcommand{\kk}{\Bbbk}
\newcommand{\Gal}{\mathrm{Gal}}

\newcommand{\Hom}{\mathrm{Hom}}
\newcommand{\projective}[1]{\mathrm{P^1}(#1)}
\newcommand{\Id}{\mathrm{Id}}
\newcommand{\Image}{\mathrm{Im} \,}
\newcommand{\Aut}[1]{\mathrm{Aut}\left(#1\right)}
\newcommand{\pr}[1]{\mathrm{pr}_{#1}}
\newcommand{\Rad}[1]{\mathrm{Rad}\left(#1\right)}
\newcommand{\Ann}[1]{\mathrm{Ann}\left(#1\right)}
\newcommand{\op}[1]{#1^{\mathrm{op}}}
\newcommand{\End}[2]{\mathrm{End}_{#2}\left(#1\right)}
\newcommand{\Ab}{\mathrm{Ab}}

\newcommand{\pp}{\mathfrak{p}}
\newcommand{\qq}{\mathfrak{q}}



% Кусочное определение функции
\newcommand{\definefuntwo}[4]{
	\begin{cases}
		#1, & #2; \\
		#3, & #4.
	\end{cases}
}

\newcommand{\definefunthree}[6]{
	\begin{cases}
		#1, & #2; \\
		#3, & #4; \\
		#5, & #6.
	\end{cases}
}

\newcommand{\definefunfour}[8]{
	\begin{cases}
		#1, & #2; \\
		#3, & #4; \\
		#5, & #6; \\
		#7, & #8.
	\end{cases}
}

\newcommand{\prob}{\operatorname{P}}
\newcommand{\events}{\mathfrak{F}}
\newcommand{\expect}{\operatorname{E}}
\newcommand{\disp}{\operatorname{D}}
\newcommand{\cov}{\operatorname{Cov}}

\newcommand{\params}{\Theta}

\newcommand{\red}[1]{{\color{red} #1}}
\newcommand{\blue}[1]{{\color{blue} #1}}

\title{Решения задач по математической статистике}
\author{Константин Зюбин}

%\pagenumbering{arabic}

\begin{document}
	
	\maketitle
	
	%	\tableofcontents
	
	%\section{Дз 2}
	
	
	
	\begin{problem}[Задача 1а]{inprocessing}
		
		Пусть $ X_1, \ldots, X_n \sim \mathcal{N}(\theta, 1) $ и $ \theta \sim \mathcal{N}(\mu, 1) $.
		Вычислим апостериорную плотность $ \theta $.
		Имеем
		$$ \pi_{\theta \mid X_1, \ldots, X_n}(t, x_1, \ldots, x_n)
		= \tfrac{f_{\theta, X_1, \ldots, X_n}(t, x_1, \ldots, x_n)}{f_{ X_1, \ldots, X_n}(x_1, \ldots, x_n)}
		= \tfrac{\frac{1}{\sqrt{2\pi}}e^{-\frac{(t - \mu)^2}{2}}\prod\limits_{i = 1}^{n} 
		\tfrac{1}{\sqrt{2\pi}}e^{-\frac{(x_i - t)^2}{2}}}{f_{ X_1, \ldots, X_n}(x_1, \ldots, x_n)} = $$
		$$ = C(x_1, \ldots, x_n, \mu)e^{-\frac{t^2 - 2t\mu}{2}-\frac{\sum\limits_{i = 1}^{n}(x_i - t)^2}{2}} 
		= C(x_1, \ldots, x_n, \mu)
		e^{-\tfrac{1}{2}\left((n + 1)t^2 - 2t\sum\limits_{i = 1}^{n} x_i - 2t\mu + \sum\limits_{i = 1}^{n} x_i^2\right)} = $$ 
		$$ = \tilde{C}(x_1, \ldots, x_n, \mu)e^{-\tfrac{n + 1}{2}\left(t - \frac{1}{n + 1}\sum\limits_{i = 1}^{n} x_i - \frac{\mu}{n + 1}\right)^2}.$$
		Тогда $ \pi_{\theta \mid X_1, \ldots, X_n} $ --- плотность случайной величины $ \hat{\theta} $ с распределением
		$$ \mathcal{N}\left(\tfrac{n\overline{X} + \mu}{n + 1}, \tfrac{1}{n + 1}\right). $$
		
	\end{problem}
	
	\begin{problem}[Задача 1б]{inprocessing}
		
		Вычислим байесовскую оценка для абсолютного риска.
		Имеем 
		$$ \tfrac{1}{2} = F_{\hat{\theta}}(q) = \Phi_{\tfrac{n\overline{X} + \mu}{n + 1}, \tfrac{1}{n + 1}}(q). $$
		Отсюда $ q - \tfrac{n\overline{X} + \mu}{n + 1} = 0 $
		и $ q = \tfrac{n\overline{X} + \mu}{n + 1} $.
		
		Вычислим байесовскую оценку для среднеквадратичного риска:
		$$ S = \expect \hat{\theta} = \tfrac{n\overline{X} + \mu}{n + 1}. $$
		
		Вычислим байесовский доверительный интервал.
		В условиях задачи имеем 
		$$ q_{\alpha} = \tfrac{z_{\alpha}}{\sqrt{n + 1}} + \tfrac{n\overline{X} + \mu}{n + 1}. $$
		Тогда доверительный интервал $ (q_{\frac{\alpha}{2}}, q_{1 - \frac{\alpha}{2}}) $ равен
		$$ \left(\tfrac{z_{\frac{\alpha}{2}}}{\sqrt{n + 1}} + \tfrac{n\overline{X} + \mu}{n + 1},
		\tfrac{z_{1 - \frac{\alpha}{2}}}{\sqrt{n + 1}} + \tfrac{n\overline{X} + \mu}{n + 1} \right). $$
		
	\end{problem}
	
	\begin{problem}[Задача 2аб]{inprocessing}
		
		Пусть $ X_1, \ldots, X_n \sim \mathcal{N}(0, \tfrac{1}{\theta}) $ и $ \theta \sim \mathrm{\Gamma}(2, \beta) $.
		Вычислим апостериорную плотность $ \theta $.
		Имеем
		$$ \pi_{\theta \mid X_1, \ldots, X_n}(t, x_1, \ldots, x_n)
		= \tfrac{f_{\theta, X_1, \ldots, X_n}(t, x_1, \ldots, x_n)}{f_{ X_1, \ldots, X_n}(x_1, \ldots, x_n)}
		= \tfrac{\frac{t}{\beta^2}e^{-\tfrac{t}{\beta}}I(t > 0)\prod\limits_{i = 1}^{n} 
			\tfrac{\sqrt{t}}{\sqrt{2\pi}}e^{-\frac{x_i^2t}{2}}}{f_{ X_1, \ldots, X_n}(x_1, \ldots, x_n)} = $$
		$$ = C(x_1, \ldots, x_n, \beta)t^{1 + \tfrac{n}{2}}e^{-\frac{t}{2}(\frac{2}{\beta} + \sum\limits_{i = 1}^{n} x_i^2)}.$$
		Тогда $ \pi_{\theta \mid X_1, \ldots, X_n} $ --- плотность случайной величины $ \hat{\theta} $ с распределением
		$$ \mathrm{\Gamma}\left(2 + \tfrac{n}{2}, \left(\tfrac{1}{\beta} + \tfrac{1}{2}S_n^2 \right)^{-1}\right), $$
		где $ S_n^2 = \sum\limits_{i = 1}^{n} X_i^2 $.
		
		Вычислим байесовскую оценку для среднеквадратичного риска:
		$$ S = \expect \hat{\theta} = \tfrac{2 + \tfrac{n}{2}}{\tfrac{1}{\beta} + \tfrac{1}{2}S_n^2}. $$
		
	\end{problem}


	\begin{problem}[Задача 3аб]{inprocessing}
		
		Пусть $ X_1, \ldots, X_n \sim \mathrm{R}[0, \theta] $ и $ \theta \sim \mathrm{B}(\alpha, 1) $.
		Вычислим апостериорную плотность $ \theta $.
		Имеем
		$$ \pi_{\theta \mid X_1, \ldots, X_n}(t, x_1, \ldots, x_n)
		= \tfrac{f_{\theta, X_1, \ldots, X_n}(t, x_1, \ldots, x_n)}{f_{ X_1, \ldots, X_n}(x_1, \ldots, x_n)}
		= \tfrac{\frac{1}{B(\alpha, 1)}t^{\alpha - 1}I(0 < t < 1)\prod\limits_{i = 1}^{n} \tfrac{1}{t}I(0 < x_i < t)}{f_{ X_1, \ldots, X_n}(x_1, \ldots, x_n)} = $$
		$$ = C(x_1, \ldots, x_n, \alpha)t^{\alpha - n - 1}I(\max\limits_{i = 1\ldots n} x_i < t < 1).$$
		Тогда $ \pi_{\theta \mid X_1, \ldots, X_n} 
		= C(X_1, \ldots, X_n, \alpha)t^{\alpha - n - 1}I(X_{(n)} < t < 1) $ --- плотность случайной величины $ \hat{\theta} $.
		Для $ C(X_1, \ldots, X_n) $ имеем выражение:
		$$ C(X_1, \ldots, X_n) 
		= \int\limits_{-\infty}^{+\infty} t^{\alpha - n - 1}I(X_{(n)} < t < 1)dt
		= \int\limits_{X_{(n)}}^{1} t^{\alpha - n - 1}dt
		= \left.t^{\alpha - n}\right|_{X_{(n)}}^{1} = 1 - X_{(n)}^{\alpha - n}. $$
		Таким образом, $$ \pi_{\theta \mid X_1, \ldots, X_n} 
		= \tfrac{t^{\alpha - n - 1}I(X_{(n)} < t < 1)}{1 - X_{(n)}^{\alpha - n}}. $$
		
		Вычислим квантили этого распределения.
		Имеем
		$$ \beta = \int\limits_{-\infty}^{q_\beta} \tfrac{t^{\alpha - n - 1}I(X_{(n)} < t < 1)}{1 - X_{(n)}^{\alpha - n}}dt
		= \int\limits_{X_{(n)}}^{q_\beta} \tfrac{t^{\alpha - n - 1}}{1 - X_{(n)}^{\alpha - n}}dt
		= \tfrac{q_\beta^{\alpha - n} - X_{(n)}^{\alpha - n}}{1 - X_{(n)}^{\alpha - n}}. $$
		Отсюда
		$$ q_\beta = \left(\beta(1 - X_{(n)}^{\alpha - n}) + X_{(n)}^{\alpha - n}\right)^{\tfrac{1}{\alpha - n}}
		= \left(\beta + (1 - \beta)X_{(n)}^{\alpha - n}\right)^{\tfrac{1}{\alpha - n}}. $$
		
		Тогда байесовский доверительный интервал для $ 1 - \beta $ есть
		$$ \left(\left(\tfrac{\beta}{2} + (1 - \tfrac{\beta}{2})X_{(n)}^{\alpha - n}\right)^{\tfrac{1}{\alpha - n}},
		\left(1 - \tfrac{\beta}{2} + \tfrac{\beta}{2}X_{(n)}^{\alpha - n}\right)^{\tfrac{1}{\alpha - n}} \right). $$
		Байесовская оценка для абсолютного риска равна
		$$ q = \left(\tfrac{1}{2} + \tfrac{1}{2}X_{(n)}^{\alpha - n}\right)^{\tfrac{1}{\alpha - n}}. $$
		
		
	\end{problem}	

	
	\begin{problem}[Задача 4аб]{inprocessing}
		
		Пусть $ T $ --- достаточная статистика. По критерию факторизации имеется разложение
		$$ L(t, x_1, \ldots, x_n) = g(t, T(x_1, \ldots, x_n))h(x_1, \ldots, x_n) $$
		для некоторых $ g $ и $ h $.
		Тогда апостериорная плотность равна
		$$ \pi_{\theta \mid X_1, \ldots, X_n}(t, x_1, \ldots, x_n)
		= \tfrac{f_{\theta, X_1, \ldots, X_n}(t, x_1, \ldots, x_n)}{f_{ X_1, \ldots, X_n}(x_1, \ldots, x_n)}
		= \tfrac{g(t, T)h(x_1, \ldots, x_n)}
		{\int\limits_{\Theta} g(t, T)h(x_1, \ldots, x_n)dt}
		= \tfrac{h(x_1, \ldots, x_n)}{\int\limits_{\Theta} g(t, T)h(x_1, \ldots, x_n)dt} \cdot g(t, T)
		= \tfrac{g(t, T)}{\int\limits_{\Theta} g(t, T)dt}. $$
		Таким образом, параметры апостериорного распределения выражаются через достаточную статистику $ T $.
		
		В задаче $ 1 $ в качестве такой статистики выступает $ \overline{X} $.
		В задаче $ 2 $ в качестве такой статистики выступает $ S_n^2 = \sum\limits X_i^2 $. 
		В задаче $ 3 $ в качестве такой статистики выступает $ X_{(n)} $.
		Все эти оценки возникают как оценки максимального правдоподобия или пропорциональные им (задача 2) и были посчитаны в 3-й домашней работе
		(такие оценки являются достаточными).
		
		
	\end{problem}
	
\end{document}