% \documentclass[12pt]{article}
\documentclass[12pt]{amsart}

\pagestyle{plain}
\usepackage[margin=2cm]{geometry} 

\usepackage{amsmath,amssymb,amsfonts,enumerate,latexsym,amsthm,textcomp,wasysym}
\usepackage{nicefrac, xfrac}
\usepackage{hyperref}
\usepackage{subfiles}
%\usepackage{tocloft}

%\usepackage{indentfirst}
\usepackage{cancel}
\usepackage{graphicx}
% \graphicspath{{pictures/}}
% \DeclareGraphicsExtensions{.pdf,.png,.jpg}
%\usepackage{russian}

% Colors
\usepackage[dvipsnames]{xcolor}
\definecolor{linkcolor}{HTML}{0000FF} % цвет ссылок
\definecolor{urlcolor}{HTML}{0000FF} % цвет гиперссылок
\definecolor{citecolor}{HTML}{0000FF} % цвет ссылки на статью
\hypersetup{pdfstartview=FitH, linkcolor=linkcolor, urlcolor=urlcolor, citecolor=citecolor, colorlinks=true}
% Пробелы, отступы и выделения
\definecolor{todocolor}{HTML}{FF4500} % цвет todo
\definecolor{defcolor}{HTML}{EE5D0F} % цвет определений
\newcommand{\TODO}[1]{\textcolor{todocolor}{НУЖНО #1}}
\renewcommand\labelenumi{\rm (\arabic{enumi})}
\renewcommand\theenumi{\rm (\arabic{enumi})}
\definecolor{completed}{HTML}{32CD32}
\definecolor{inprocessing}{HTML}{D19A0F}

% Pictures and diagrams
\usepackage[matrix, arrow, curve]{xy} 
\usepackage{tikz-cd}
\usepackage{tikz}
\usetikzlibrary{shapes.geometric}
\usepackage{makecell}

\tikzset{
	symbol/.style={
		draw=none,
		every to/.append style={
			edge node={node [sloped, allow upside down, auto=false]{$#1$}}}
	}
}

\usepackage[utf8]{inputenc}
\usepackage[russian]{babel}
\usepackage{verbatim}
\makeatletter
\def\@settitle{\begin{center}%
		\baselineskip14\p@\relax
		\bfseries
		\large \@title
	\end{center}%
}
\makeatother


%%%%%%%%%%%%%%%%%%%%%%%%%%%%%%%%%%%%%%%%%%%%%%%%%%%%%%%%%%%%
% % commands for making comments
%\usepackage[dvipsnames]{xcolor}
\newcommand{\YP}[1]{\footnote{\textcolor{red}{YP: #1}}}
\newcommand{\yp}[1]{\leavevmode{\color{red}{#1}}}
% {\textcolor{orange}{#1}} 
\usepackage[normalem]{ulem}
%%%%%%%%%%%%%%%%%%%%%%%%%%%%%%%%%%%%%%%%%%%%%%%%%%%%%%%%%%%%

% \textheight=270mm
% \textwidth=190mm
% \voffset=-40mm
% \hoffset=-35mm
% \pagestyle{empty}
% 
% \\SLoppy

\emergencystretch=5pt

\setcounter{tocdepth}{4}
\setcounter{secnumdepth}{4}

% Theorems

\newtheorem{theorem}{Теорема}
\newtheorem*{definition}{Определение}
\newtheorem{proposition}[theorem]{Предложение}
\newtheorem{lemma}[theorem]{Лемма}
\newtheorem{corollary}[theorem]{Следствие}
\newtheorem*{remark*}{Замечание}


\theoremstyle{definition}

% Environments

\newenvironment{problem}[2][Problem name]{\indent \textcolor{#2}{\textbf{#1}} \indent}{\indent}
\newenvironment{squarestatement}[1][Statement]{\indent \textbf{[#1]} \indent}
{$ \hfill \lhd $ \indent}

% New Commands

% Set definition
\newcommand{\defineset}[2]{\left\{
	\left.
	#1 \
	\right\vert
	#2
	\right\}}

\newcommand{\Alt}{\mathfrak{A}}
\newcommand{\Sym}{\mathfrak{S}}
\newcommand{\D}{\mathrm{D}}
\newcommand{\Q}{\mathrm{Q}}
\newcommand{\rC}{\mathrm{C}}
\newcommand{\T}{\mathrm{T}}
\newcommand{\rO}{\mathrm{O}}
\newcommand{\I}{\mathrm{I}}
\newcommand{\CC}{\mathbb{C}}
\newcommand{\RR}{\mathbb{R}}
\newcommand{\FF}{\mathbb{F}}
\newcommand{\EE}{\mathbb{E}}

\newcommand{\calA}{\mathcal{A}}
\newcommand{\calB}{\mathcal{B}}
\newcommand{\calE}{\mathcal{E}}
\newcommand{\calP}{\mathcal{P}}

\newcommand{\GL}{\operatorname{GL}}
\newcommand{\SL}{\operatorname{SL}}
\newcommand{\PGL}{\operatorname{PGL}}
\newcommand{\PSL}{\operatorname{PSL}}
\newcommand{\SU}{\operatorname{SU}}
\newcommand{\SO}{\operatorname{SO}}
\newcommand{\diag}{\operatorname{diag}}
\newcommand{\characteristic}{\operatorname{char}}
\newcommand{\kk}{\Bbbk}
\newcommand{\Gal}{\mathrm{Gal}}

\newcommand{\Hom}{\mathrm{Hom}}
\newcommand{\projective}[1]{\mathrm{P^1}(#1)}
\newcommand{\Id}{\mathrm{Id}}
\newcommand{\Image}{\mathrm{Im} \,}
\newcommand{\Aut}[1]{\mathrm{Aut}\left(#1\right)}
\newcommand{\pr}[1]{\mathrm{pr}_{#1}}
\newcommand{\Rad}[1]{\mathrm{Rad}\left(#1\right)}
\newcommand{\Ann}[1]{\mathrm{Ann}\left(#1\right)}
\newcommand{\op}[1]{#1^{\mathrm{op}}}
\newcommand{\End}[2]{\mathrm{End}_{#2}\left(#1\right)}
\newcommand{\Ab}{\mathrm{Ab}}

\newcommand{\grad}{\operatorname{grad}}

\newcommand{\vect}[1]{\overrightarrow{#1}}

\newcommand{\pp}{\mathfrak{p}}
\newcommand{\qq}{\mathfrak{q}}

\newcommand{\spmatrix}[4]{
	\left( \begin{smallmatrix}
		#1 & #2 \\
		#3 & #4
	\end{smallmatrix} \right)
}


% Кусочное определение функции
\newcommand{\definefuntwo}[4]{
	\begin{cases}
		#1, & #2; \\
		#3, & #4.
	\end{cases}
}

\newcommand{\definefunthree}[6]{
	\begin{cases}
		#1, & #2; \\
		#3, & #4; \\
		#5, & #6.
	\end{cases}
}

\newcommand{\definefunfour}[8]{
	\begin{cases}
		#1, & #2; \\
		#3, & #4; \\
		#5, & #6; \\
		#7, & #8.
	\end{cases}
}

\newcommand{\prob}{\operatorname{P}}
\newcommand{\events}{\mathfrak{F}}
\newcommand{\expect}{\operatorname{E}}
\newcommand{\disp}{\operatorname{D}}
\newcommand{\cov}{\operatorname{Cov}}

\newcommand{\params}{\Theta}

\newcommand{\red}[1]{{\color{red} #1}}
\newcommand{\blue}[1]{{\color{blue} #1}}

\title{Решения задач по математической статистике}
\author{Константин Зюбин}

%\pagenumbering{arabic}

\begin{document}
	
	\maketitle
	
	%	\tableofcontents
	
	%\section{Дз 2}
	
	\begin{problem}[Задача 1аб]{inprocessing}
		
		Пусть $ X_1, \ldots, X_n \sim \mathcal{N}(\theta_x, \theta^2) $
		и $ Y_1, \ldots, Y_n \sim \mathcal{N}(\theta_y, \theta^2) $ --- независимые случайные величины.
		
		Пусть нулевая гипотеза предполагает истинность неравенства $ \theta_x \geqslant \theta_y $.
		
		Вычислим функцию правдоподобия и найдём её точку максимума.
		$$ L(\theta_x, \theta_y, \theta, \vect{X}, \vect{Y})
		= \prod\limits_{i = 1}^{n} \tfrac{1}{2\pi \theta^2}e^{-\tfrac{(X_i - \theta_x)^2 + (Y_i - \theta_y)^2}{2\theta^2}}; $$
		$$ \ln L(\theta_x, \theta_y, \theta, \vect{X}, \vect{Y})
		= -n\ln 2\pi - 2n\ln \theta - \tfrac{1}{2\theta^2}\sum\limits_{i = 1}^{n} \left((X_i - \theta_x)^2 + (Y_i - \theta_y)^2\right). $$
		Частные производные равны
		$$ \tfrac{\partial}{\partial \theta_x} \ln L(\theta_x, \theta_y, \theta, \vect{X}, \vect{Y})
		= \tfrac{1}{\theta^2}\left(-n\theta_x + \sum\limits_{i = 1}^{n} X_i\right); $$
		$$ \tfrac{\partial}{\partial \theta_y} \ln L(\theta_x, \theta_y, \theta, \vect{X}, \vect{Y})
		= \tfrac{1}{\theta^2}\left(-n\theta_y + \sum\limits_{i = 1}^{n} Y_i\right); $$
		$$ \tfrac{\partial}{\partial \theta} \ln L(\theta_x, \theta_y, \theta, \vect{X}, \vect{Y})
		= -\tfrac{2n}{\theta} + \tfrac{1}{\theta^3}\sum\limits_{i = 1}^{n} \left((X_i - \theta_x)^2 + (Y_i - \theta_y)^2\right). $$
		Точка максимума имеет координаты
		$ \hat \theta_x = \overline{X}, \hat \theta_y = \overline{Y},
		\hat \theta^2 = \tfrac{1}{2n}\sum\limits_{i = 1}^{n} \left((X_i - \overline{X})^2 + (Y_i - \overline{Y})^2\right) $.
		Значение в точке максимума равно
		$$ \tfrac{e^{-n}}{2^n\pi^n\hat\theta^{2n}}. $$
		
		Если $ \overline{X} > \overline{Y} $, то точка максимума при условии гипотезы оказывается такой же, 
		как и в общем случае. Поэтому будем рассматривать случай, когда $ \overline{X} \leqslant \overline{Y} $.
		Тогда максимум достигается на границе $ \theta_x = \theta_y =: \theta_{xy} $.
		Вычислим точку максимума и его значение.
		Имеем
		$$ L(\theta_0, \theta, \vect{X}, \vect{Y})
		= \prod\limits_{i = 1}^{n} \tfrac{1}{2\pi \theta^2}e^{-\tfrac{(X_i - \theta_{xy})^2 + (Y_i - \theta_{xy})^2}{2\theta^2}}; $$
		$$ \ln L(\theta_0, \theta, \vect{X}, \vect{Y})
		= -n\ln 2\pi - 2n\ln \theta - \tfrac{1}{2\theta^2}\sum\limits_{i = 1}^{n} \left((X_i - \theta_0)^2 + (Y_i - \theta_0)^2\right). $$
		Частные производные равны
		$$ \tfrac{\partial}{\partial \theta_x} \ln L(\theta_x, \theta_y, \theta, \vect{X}, \vect{Y})
		= \tfrac{1}{2\theta^2}\left(-2n\theta_{xy} + \sum\limits_{i = 1}^{n} X_i + \sum\limits_{i = 1}^{n} Y_i\right); $$
		$$ \tfrac{\partial}{\partial \theta} \ln L(\theta_x, \theta_y, \theta, \vect{X}, \vect{Y})
		= -\tfrac{2n}{\theta} + \tfrac{1}{\theta^3}\sum\limits_{i = 1}^{n} \left((X_i - \theta_{xy})^2 + (Y_i - \theta_{xy})^2\right). $$
		Точка максимума имеет координаты
		$ \hat \theta_{xy} = \tfrac{\overline{X} + \overline{Y}}{2},
		\hat \theta^2_{H_0} = \tfrac{1}{2n}\sum\limits_{i = 1}^{n} 
		\left((X_i - \tfrac{\overline{X} + \overline{Y}}{2})^2 + (Y_i - \tfrac{\overline{X} + \overline{Y}}{2})^2\right) $.
		Значение в точке максимума равно
		$$ \tfrac{e^{-n}}{2^n\pi^n\hat\theta_{H_0}^{2n}}. $$
		Отсюда 
		$$ \lambda = \tfrac{\sup_{\theta \in \Theta} L(\theta_x, \theta_y, \theta, \vect{X}, \vect{Y})}
		{\sup_{\theta \in \Theta_0} L(\theta_0, \theta, \vect{X}, \vect{Y})}
		= \tfrac{\hat\theta_{H_0}^{2n}}{\hat\theta^{2n}}. $$
		$$ \sqrt[n]{\lambda} = \tfrac{\sum\limits_{i = 1}^{n} 
			\left((X_i - \tfrac{\overline{X} + \overline{Y}}{2})^2 + (Y_i - \tfrac{\overline{X} + \overline{Y}}{2})^2\right)}
		{\sum\limits_{i = 1}^{n} \left((X_i - \overline{X})^2 + (Y_i - \overline{Y})^2\right)}
		= \tfrac{\sum\limits_{i = 1}^{n} 
			\left((X_i - \overline{X} - \tfrac{\overline{Y} - \overline{X}}{2})^2 + (Y_i - \overline{Y} - \tfrac{\overline{X} - \overline{Y}}{2})^2\right)}
		{\sum\limits_{i = 1}^{n} \left((X_i - \overline{X})^2 + (Y_i - \overline{Y})^2\right)} 
		= 1 +
		\tfrac{2n(\tfrac{\overline{X} - \overline{Y}}{2})^2}
		{\sum\limits_{i = 1}^{n} \left((X_i - \overline{X})^2 + (Y_i - \overline{Y})^2\right)}. $$
		Отсюда
		$$ \sqrt[n]{\lambda} - 1 = \tfrac{n}{2} \cdot \tfrac{(\overline{X} - \overline{Y})^2}
		{\sum\limits_{i = 1}^{n} \left((X_i - \overline{X})^2 + (Y_i - \overline{Y})^2\right)}. $$
		Имеем следующие распределения для случайных величин
		$$ (\overline{X} - \overline{Y}) - (\theta_x - \theta_y) \sim \mathcal{N}(0, \tfrac{2\theta^2}{n}); $$
		$$ \tfrac{1}{\theta^2}\sum\limits_{i = 1}^{n} (X_i - \overline{X})^2 \sim \xi_{n - 1}^2; $$
		$$ \tfrac{1}{\theta^2}\sum\limits_{i = 1}^{n} (X_i - \overline{X})^2
		+ \tfrac{1}{\theta^2}\sum\limits_{i = 1}^{n} (Y_i - \overline{Y})^2 \sim \xi_{2n - 2}^2; $$
		Тогда
		$$ \sqrt{\sqrt[n]{\lambda} - 1} \cdot 
		\sqrt{\tfrac{2}{n} \cdot \tfrac{\frac{n}{2\theta^2}}{\frac{1}{(2n - 2)\theta^2}}}
		= \tfrac{\sqrt{\frac{n}{2\theta^2}}((\overline{X} - \overline{Y}) - (\theta_x - \theta_y))}
		{\sqrt{\frac{1}{(2n - 2)\theta^2}\sum\limits_{i = 1}^{n} \left((X_i - \overline{X})^2 + (Y_i - \overline{Y})^2\right)}}
		\sim t_{2n - 2} $$
		и
		$$ \sqrt{\sqrt[n]{\lambda} - 1} \cdot \tfrac{1}{\sqrt{2n - 2}} \sim t_{2n - 2}. $$
		
		Пусть $ q_{\alpha} $ --- квантиль уровня $ \alpha $ для распределения Стьюдента $ t_{2n - 2} $.
		Будем отклонять гипотезу при $ \tfrac{\sqrt{\frac{n}{2}}(\overline{X} - \overline{Y})}
		{\sqrt{\tfrac{1}{2n - 2}\sum\limits_{i = 1}^{n} \left((X_i - \overline{X})^2 + (Y_i - \overline{Y})^2\right)}} < q_{\alpha} $. 
		
		Обозначим через $ F_{t_{2n - 2}} $ --- функцию распределения Стьюдента.
		Тогда
		$$ \sup\limits_{\theta_x \geqslant \theta_y} 
		\prob\left(
		\tfrac{\sqrt{2n}(\tfrac{\overline{X} - \overline{Y}}{2})}
		{\sqrt{\tfrac{1}{2n - 2}\sum\limits_{i = 1}^{n} 
				\left(
				(X_i - \overline{X})^2 + (Y_i - \overline{Y})^2
				\right)
			}
		}
		< q_{\alpha}
		\right) = $$  
		$$
		= \sup\limits_{\theta_x \geqslant \theta_y} 
		\prob\left(
		\tfrac{\sqrt{\tfrac{n}{2}}((\overline{X} - \overline{Y}) - (\theta_x - \theta_y))}
		{\sqrt{\tfrac{1}{2n - 2}
				\sum\limits_{i = 1}^{n} 
				\left(
				(X_i - \overline{X})^2 + (Y_i - \overline{Y})^2
				\right)
			}
		}
		< q_{\alpha} 
		- \tfrac{\sqrt{\tfrac{n}{2}}(\theta_x - \theta_y)}
		{\sqrt{\tfrac{1}{2n - 2}
				\sum\limits_{i = 1}^{n} 
				\left(
				(X_i - \overline{X})^2 + (Y_i - \overline{Y})^2
				\right)
			}
		} 
		\right) = $$ 
		$$
		= \sup\limits_{\theta_x \geqslant \theta_y} 
		F_{t_{2n - 2}}\left(
		q_{\alpha} 
		- \tfrac{\sqrt{\tfrac{n}{2}}(\theta_x - \theta_y)}
		{\sqrt{\tfrac{1}{2n - 2}
				\sum\limits_{i = 1}^{n} 
				\left(
				(X_i - \overline{X})^2 + (Y_i - \overline{Y})^2
				\right)
			}
		}
		\right)
		= F_{t_{2n - 2}}(q_\alpha - 0) = \alpha. $$
		Аналогично, имеем выражение для мощности:
		$$ \beta(\theta_1, \theta_2) =
		F_{t_{2n - 2}}\left(
		q_{\alpha} 
		- \tfrac{\sqrt{\tfrac{n}{2}}(\theta_x - \theta_y)}
		{\sqrt{\tfrac{1}{2n - 2}
				\sum\limits_{i = 1}^{n} 
				\left(
				(X_i - \overline{X})^2 + (Y_i - \overline{Y})^2
				\right)
			}
		}
		\right), $$
		что больше $ F_{t_{2n - 2}}(q_\alpha) = \alpha $ при $ \theta_x < \theta_y $.
		
	\end{problem}
	
	\begin{problem}[Задача 2]{inprocessing}
		
		Рассмотрим независимые случайные величины $ X_1, \ldots, X_n \sim \mathcal{N}(\mu_x, \theta_x^2) $
		и $ Y_1, \ldots \mathcal{N}(\mu_y, \theta_y^2) $. Параметры $ \mu_x, \mu_y $ будем считать известными числами.
		
		Пусть гипотеза состоит в том, что $ \theta_x \geqslant 2\theta_y $.
		Построим точный критерий обобщённого отношения правдоподобий.
		Имеем
		$$ L(\theta_x, \theta_y, \vect X, \vect Y) = \prod\limits_{i = 1}^{n} 
		\tfrac{1}{2\pi \theta_x\theta_y} e^{-\tfrac{(X_i - \mu_x)^2}{2\theta_x^2} - \tfrac{(Y_i - \mu_y)^2}{2\theta_y^2}}, $$
		$$ \ln L(\theta_x, \theta_y, \vect X, \vect Y) = -n\ln 2\pi - n\ln \theta_x - n\ln \theta_y - \tfrac{1}{2\theta_x^2}\sum\limits_{i = 1}^{n} (X_i - \mu_x)^2 - \tfrac{1}{2\theta_y^2}\sum\limits_{i = 1}^{n} (Y_i - \mu_y)^2. $$
		Тогда
		$$ \tfrac{\partial}{\partial \theta_x} \ln L(\theta_x, \theta_y, \vect X, \vect Y) = -\tfrac{n}{\theta_x} 
		+ \tfrac{1}{\theta_x^3}\sum\limits_{i = 1}^{n} (X_i - \mu_x)^2, $$
		$$ \tfrac{\partial}{\partial \theta_y} \ln L(\theta_x, \theta_y, \vect X, \vect Y) = -\tfrac{n}{\theta_y} 
		+ \tfrac{1}{\theta_y^3}\sum\limits_{i = 1}^{n} (Y_i - \mu_y)^2. $$
		Максимум достигается в точке с координатами 
		$ S_x^2 := \hat \theta_x^2 = \tfrac{1}{n}\sum\limits_{i = 1}^{n} (X_i - \mu_x)^2 $
		и $ S_y^2 := \hat \theta_y^2 = \tfrac{1}{n}\sum\limits_{i = 1}^{n} (Y_i - \mu_y)^2 $.
		
		Заметим, что
		$$ \tfrac{S_x^2}{S_y^2} = \tfrac{\sum\limits_{i = 1}^{n} (X_i - \mu_x)^2}{\sum\limits_{i = 1}^{n} (Y_i - \mu_y)^2}
		= \tfrac{\frac{1}{n\theta_x^2}\sum\limits_{i = 1}^{n} (X_i - \mu_x)^2}{\frac{1}{n\theta_y^2}\sum\limits_{i = 1}^{n} (Y_i - \mu_y)^2} \cdot \tfrac{\theta_x^2}{\theta_y^2} $$
		и $ \tfrac{\frac{1}{n\theta_x^2}\sum\limits_{i = 1}^{n} (X_i - \mu_x)^2}{\frac{1}{n\theta_y^2}\sum\limits_{i = 1}^{n} (Y_i - \mu_y)^2} \sim F_{n, n} $.
		
		Если $ S_x^2 \geqslant 4S_y^2 $, то точка максимума лежит в подмножестве $ \{\theta_x > 2\theta_y\} $
		и поэтому отношение супремумов правдоподобий будет равно 1. Будем рассматривать случай, когда $ S_x^2 < 4S_y^2 $.
		Тогда максимум достигается на границе области $ \theta_x = 2\theta_y $, $ \theta_y =: \theta_{xy} $. 
		Найдём его
		$$ L(\theta_{xy}, \vect X, \vect Y) = \prod\limits_{i = 1}^{n} 
		\tfrac{1}{4\pi \theta_{xy}^2} e^{-\tfrac{(X_i - \mu_x)^2}{8\theta_{xy}^2} - \tfrac{(Y_i - \mu_y)^2}{2\theta_{xy}^2}}, $$
		$$ \ln L(\theta_{xy}, \vect X, \vect Y) = -n\ln 4\pi - 2n\ln \theta_{xy} - \tfrac{1}{8\theta_{xy}^2}\sum\limits_{i = 1}^{n} (X_i - \mu_x)^2 - \tfrac{1}{2\theta_{xy}^2}\sum\limits_{i = 1}^{n} (Y_i - \mu_y)^2. $$
		Тогда
		$$ \tfrac{\partial}{\partial \theta_{xy}} \ln L(\theta_x, \theta_y, \vect X, \vect Y) 
		= -\tfrac{2n}{\theta_{xy}} 
		+ \tfrac{1}{\theta_{xy}^3}\sum\limits_{i = 1}^{n}( \tfrac{1}{4}(X_i - \mu_x)^2 + (Y_i - \mu_y)^2). $$
		Точка максимума есть $ \hat \theta_{xy}^2 = \tfrac{1}{2n}\sum\limits_{i = 1}^{n}( \tfrac{1}{4}(X_i - \mu_x)^2 + (Y_i - \mu_y)^2) = \tfrac{1}{8}S_x^2 + \tfrac{1}{2}S_y^2. $
		
		Имеем
		$$ \lambda = \tfrac{\sup\limits_{\vect \theta \in \Theta} L(\theta_x, \theta_y, \vect X, \vect Y)}
		{\sup\limits_{\vect \theta \in \Theta_0} L(\theta_x, \theta_y, \vect X, \vect Y)}
		= \left(\tfrac{\hat \theta_{xy}^2}{S_xS_y} \right)^n
		= \left(\tfrac{\tfrac{1}{8}S_x^2 + \tfrac{1}{2}S_y^2}{S_xS_y} \right)^n
		= \left(\tfrac{1}{8}\tfrac{S_x}{S_y} + \tfrac{1}{2}(\tfrac{S_x}{S_y})^{-1}\right)^n, $$
		$$ \sqrt[n]{\lambda}
		= \tfrac{1}{8}\tfrac{S_x}{S_y} + \tfrac{1}{2}(\tfrac{S_x}{S_y})^{-1}. $$
		По предположению $ 0 < \tfrac{S_x}{S_y} < 2 $.
		Рассмотрим функцию $ f(u) = \tfrac{u}{8} + \tfrac{1}{2u} $. Её производная равна $ f'(u) = \tfrac{1}{8} - \tfrac{1}{2u^2} $
		и она меньше нуля при $ 0 < u < 2 $. Следовательно, функция $ f $ монотонно убывает на $ (0, 2) $.
 		
		Имеем
		$$ \prob\left(\tfrac{S_x^2}{S_y^2} < c_\alpha \right)
		= \prob\left(\tfrac{S_x^2}{S_y^2} \cdot \tfrac{\theta_y^2}{\theta_x^2} 
		< \tfrac{c_\alpha\theta_y^2}{\theta_x^2} \right) 
		= F_{F_{2n, 2n}}(\tfrac{c_\alpha\theta_y^2}{\theta_x^2}). $$
	
		Будет отвергать гипотезу при $ \tfrac{S_x^2}{S_y^2} <
		 4q_\alpha = c_\alpha $, где $ q_{\alpha} $ --- квантиль уровня $ \alpha $
		распределения Фишера-Снедекора.
		Тогда можно вычислить уровень значимости
		$$ \sup\limits_{\theta_x \geqslant 2\theta_y} \prob\left(\tfrac{S_x^2}{S_y^2} < c_\alpha \right)
		= \sup\limits_{\theta_x \geqslant 2\theta_y} F_{F_{2n, 2n}}(\tfrac{2\theta_y^2}{\theta_x^2} \cdot q_{\alpha})
		= \alpha $$
		и мощность критерия:
		$$ \beta(\theta_x, \theta_y) = \prob\left(\tfrac{S_x^2}{S_y^2} < c_\alpha \right)
		= F_{F_{2n, 2n}}(\tfrac{4\theta_y^2}{\theta_x^2} \cdot q_{\alpha})
		> \alpha. $$
	\end{problem}
	
\end{document}

\begin{problem}[Задача 3]{inprocessing}
	
	Предположим, что в предыдущей задаче $ \mu_x $ и $ \mu_y $ являются параметрами.
	Построим точный критерий обобщённого правдоподобия для гипотезы $ 
	\theta_x \leqslant 2\theta_y $.
	
	Точкой максимума для $ L(\theta_x, \theta_y, \mu_x, \mu_y, \vect X, \vect Y) $ 
	будет точка с координатами $ \hat \mu_x = \overline{X}, \hat \mu_y = \overline{Y} $
	и $ S_x^2 = \hat \theta_x^2 = \tfrac{1}{n}\sum\limits_{i = 1}^{n} (X_i - \overline{X})^2,
	S_y^2 = \hat \theta_y^2 = \tfrac{1}{n}\sum\limits_{i = 1}^{n} (Y_i - \overline{Y})^2 $.
	
	В этом случае
	$$ \tfrac{S_x^2}{S_y^2} 
	= \frac{\frac{1}{(n - 1)\theta_x^2}\sum\limits_{i = 1}^{n} (X_i - \overline{X})^2}
	{\frac{1}{(n - 1)\theta_y^2}\sum\limits_{i = 1}^{n} (Y_i - \overline{Y})^2} $$
	
	
	При условии гипотезы, если $ \hat\theta_y^2 $
	
\end{problem}