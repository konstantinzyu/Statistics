% \documentclass[12pt]{article}
\documentclass[12pt]{amsart}

\pagestyle{plain}
\usepackage[margin=2cm]{geometry} 

\usepackage{amsmath,amssymb,amsfonts,enumerate,latexsym,amsthm,textcomp,wasysym}
\usepackage{hyperref}
\usepackage{subfiles}
%\usepackage{tocloft}

%\usepackage{indentfirst}
\usepackage{cancel}
\usepackage{graphicx}
% \graphicspath{{pictures/}}
% \DeclareGraphicsExtensions{.pdf,.png,.jpg}
%\usepackage{russian}

% Colors
\usepackage[dvipsnames]{xcolor}
\definecolor{linkcolor}{HTML}{0000FF} % цвет ссылок
\definecolor{urlcolor}{HTML}{0000FF} % цвет гиперссылок
\definecolor{citecolor}{HTML}{0000FF} % цвет ссылки на статью
\hypersetup{pdfstartview=FitH, linkcolor=linkcolor, urlcolor=urlcolor, citecolor=citecolor, colorlinks=true}
% Пробелы, отступы и выделения
\definecolor{todocolor}{HTML}{FF4500} % цвет todo
\definecolor{defcolor}{HTML}{EE5D0F} % цвет определений
\newcommand{\TODO}[1]{\textcolor{todocolor}{НУЖНО #1}}
\renewcommand\labelenumi{\rm (\arabic{enumi})}
\renewcommand\theenumi{\rm (\arabic{enumi})}
\definecolor{completed}{HTML}{32CD32}
\definecolor{inprocessing}{HTML}{D19A0F}

% Pictures and diagrams
\usepackage[matrix, arrow, curve]{xy} 
\usepackage{tikz-cd}
\usepackage{tikz}
\usetikzlibrary{shapes.geometric}
\usepackage{makecell}

\tikzset{
	symbol/.style={
		draw=none,
		every to/.append style={
			edge node={node [sloped, allow upside down, auto=false]{$#1$}}}
	}
}

\usepackage[utf8]{inputenc}
\usepackage[russian]{babel}
\usepackage{verbatim}
\makeatletter
\def\@settitle{\begin{center}%
		\baselineskip14\p@\relax
		\bfseries
		\large \@title
	\end{center}%
}
\makeatother


%%%%%%%%%%%%%%%%%%%%%%%%%%%%%%%%%%%%%%%%%%%%%%%%%%%%%%%%%%%%
% % commands for making comments
%\usepackage[dvipsnames]{xcolor}
\newcommand{\YP}[1]{\footnote{\textcolor{red}{YP: #1}}}
\newcommand{\yp}[1]{\leavevmode{\color{red}{#1}}}
% {\textcolor{orange}{#1}} 
\usepackage[normalem]{ulem}
%%%%%%%%%%%%%%%%%%%%%%%%%%%%%%%%%%%%%%%%%%%%%%%%%%%%%%%%%%%%

% \textheight=270mm
% \textwidth=190mm
% \voffset=-40mm
% \hoffset=-35mm
% \pagestyle{empty}
% 
% \\SLoppy

\emergencystretch=5pt

\setcounter{tocdepth}{4}
\setcounter{secnumdepth}{4}

% Theorems

\newtheorem{theorem}{Теорема}
\newtheorem*{definition}{Определение}
\newtheorem{proposition}[theorem]{Предложение}
\newtheorem{lemma}[theorem]{Лемма}
\newtheorem{corollary}[theorem]{Следствие}
\newtheorem*{remark*}{Замечание}


\theoremstyle{definition}

% Environments

\newenvironment{problem}[2][Problem name]{\indent \textcolor{#2}{\textbf{#1}} \indent}{\indent}
\newenvironment{squarestatement}[1][Statement]{\indent \textbf{[#1]} \indent}
{$ \hfill \lhd $ \indent}

% New Commands

% Set definition
\newcommand{\defineset}[2]{\left\{
	\left.
	#1 \
	\right\vert
	#2
	\right\}}

\newcommand{\Alt}{\mathfrak{A}}
\newcommand{\Sym}{\mathfrak{S}}
\newcommand{\D}{\mathrm{D}}
\newcommand{\Q}{\mathrm{Q}}
\newcommand{\rC}{\mathrm{C}}
\newcommand{\T}{\mathrm{T}}
\newcommand{\rO}{\mathrm{O}}
\newcommand{\I}{\mathrm{I}}
\newcommand{\CC}{\mathbb{C}}
\newcommand{\RR}{\mathbb{R}}
\newcommand{\FF}{\mathbb{F}}
\newcommand{\EE}{\mathbb{E}}

\newcommand{\calA}{\mathcal{A}}
\newcommand{\calB}{\mathcal{B}}
\newcommand{\calE}{\mathcal{E}}
\newcommand{\calP}{\mathcal{P}}

\newcommand{\GL}{\operatorname{GL}}
\newcommand{\SL}{\operatorname{SL}}
\newcommand{\PGL}{\operatorname{PGL}}
\newcommand{\PSL}{\operatorname{PSL}}
\newcommand{\SU}{\operatorname{SU}}
\newcommand{\SO}{\operatorname{SO}}
\newcommand{\diag}{\operatorname{diag}}
\newcommand{\characteristic}{\operatorname{char}}
\newcommand{\kk}{\Bbbk}
\newcommand{\Gal}{\mathrm{Gal}}

\newcommand{\Hom}{\mathrm{Hom}}
\newcommand{\projective}[1]{\mathrm{P^1}(#1)}
\newcommand{\Id}{\mathrm{Id}}
\newcommand{\Image}{\mathrm{Im} \,}
\newcommand{\Aut}[1]{\mathrm{Aut}\left(#1\right)}
\newcommand{\pr}[1]{\mathrm{pr}_{#1}}
\newcommand{\Rad}[1]{\mathrm{Rad}\left(#1\right)}
\newcommand{\Ann}[1]{\mathrm{Ann}\left(#1\right)}
\newcommand{\op}[1]{#1^{\mathrm{op}}}
\newcommand{\End}[2]{\mathrm{End}_{#2}\left(#1\right)}
\newcommand{\Ab}{\mathrm{Ab}}

\newcommand{\pp}{\mathfrak{p}}
\newcommand{\qq}{\mathfrak{q}}



% Кусочное определение функции
\newcommand{\definefuntwo}[4]{
	\begin{cases}
		#1, & #2; \\
		#3, & #4.
	\end{cases}
}

\newcommand{\definefunthree}[6]{
	\begin{cases}
		#1, & #2; \\
		#3, & #4; \\
		#5, & #6.
	\end{cases}
}

\newcommand{\definefunfour}[8]{
	\begin{cases}
		#1, & #2; \\
		#3, & #4; \\
		#5, & #6; \\
		#7, & #8.
	\end{cases}
}

\newcommand{\prob}{\operatorname{P}}
\newcommand{\events}{\mathfrak{F}}
\newcommand{\expect}{\operatorname{E}}
\newcommand{\disp}{\operatorname{D}}
\newcommand{\cov}{\operatorname{Cov}}

\newcommand{\params}{\Theta}

\newcommand{\red}[1]{{\color{red} #1}}
\newcommand{\blue}[1]{{\color{blue} #1}}

\title{Решения задач по математической статистике}
\author{Константин Зюбин}

%\pagenumbering{arabic}

\begin{document}
	
	\maketitle
	
	%	\tableofcontents
	
	%\section{Дз 2}
	
	\begin{problem}[Задача 1 ($ \mathcal{N}(\theta, \theta) $)]{inprocessing}
		
		Построим асимптотические доверительные интервалы в модели $ X_1, \ldots, X_n \sim \mathcal{N}(\theta, \theta) $
		на основе оценки максимального правдоподобия $ T = \tfrac{-1 + \sqrt{1 + 4\overline{X^2}}}{2} $ (см. домашнюю работу 3).
		Эта оценка состоятельна для $ \theta $ и имеет асимптотическую дисперсию $ 2\theta^2 $ (см. домашнюю работу 4).
		Тогда оценка $ \tfrac{-1 + \sqrt{1 + 4\overline{X^2}}}{\sqrt{2}} $ состоятельна для $ \sqrt{2}\theta $.
		$$ \prob_\theta(a \leqslant \tfrac{\sqrt{n}(T - \theta)}{\sqrt{2}\theta}
		\leqslant b) \underset{n \to +\infty}{\to} \Phi(b) - \Phi(a), $$
		$$ \prob_\theta(a \leqslant \tfrac{\sqrt{n}(T - \theta)}{\sqrt{2}T}
		\leqslant b) \underset{n \to +\infty}{\to} \Phi(b) - \Phi(a), $$
		$$ \prob_\theta(T(1 - \tfrac{b\sqrt{2}}{\sqrt{n}}) \leqslant \theta
		\leqslant T(1 - \tfrac{a\sqrt{2}}{\sqrt{n}})) \underset{n \to +\infty}{\to} \Phi(b) - \Phi(a). $$
		
		Для $ \Phi(b) - \Phi(a) = 1 - \alpha $ положим $ a = z_{\frac{\alpha}{2}}, b = z_{1 - \frac{\alpha}{2}} $.
		Тогда доверительным будет интервал $ (T(1 - \tfrac{z_{1 - \frac{\alpha}{2}}\sqrt{2}}{\sqrt{n}}), T(1 - \tfrac{z_{\frac{\alpha}{2}}\sqrt{2}}{\sqrt{n}})) $.
		
	\end{problem}
	
	\begin{problem}[Задача 1 ($ \mathcal{N}(\theta, \theta^2) $)]{inprocessing}
		
		Построим асимптотические доверительные интервалы в модели $ X_1, \ldots, X_n \sim \mathcal{N}(\theta, \theta^2) $
		на основе оценки максимального правдоподобия 
		$ T = \tfrac{-\overline{X} \pm \sqrt{\overline{X}^2 + 4\overline{X^2}}}{2} $ (см. домашнюю работу 3).
		Эта оценка состоятельна для $ \theta $ и имеет асимптотическую дисперсию $ \tfrac{\theta^2}{3} $ (см. домашнюю работу 4).
		Тогда оценка $ \tfrac{T}{\sqrt{3}} $ состоятельна для $ \tfrac{\theta}{\sqrt{3}} $.
		$$ \prob_\theta(a \leqslant \tfrac{\sqrt{3n}(T - \theta)}{\theta}
		\leqslant b) \underset{n \to +\infty}{\to} \Phi(b) - \Phi(a), $$
		$$ \prob_\theta(a \leqslant \tfrac{\sqrt{3n}(T - \theta)}{T}
		\leqslant b) \underset{n \to +\infty}{\to} \Phi(b) - \Phi(a), $$
		$$ \prob_\theta(T(1 - \tfrac{b}{\sqrt{3n}}) \leqslant \theta
		\leqslant T(1 - \tfrac{a}{\sqrt{3n}})) \underset{n \to +\infty}{\to} \Phi(b) - \Phi(a). $$
		
		Для $ \Phi(b) - \Phi(a) = 1 - \alpha $ положим $ a = z_{\frac{\alpha}{2}}, b = z_{1 - \frac{\alpha}{2}} $.
		Тогда доверительным будет интервал $ (T(1 - \tfrac{z_{1 - \frac{\alpha}{2}}}{\sqrt{3n}}), T(1 - \tfrac{z_{\frac{\alpha}{2}}}{\sqrt{3n}})) $.
	\end{problem}
	
	\begin{problem}[Задача 1 ($ \mathcal{N}(0, \theta) $)]{inprocessing}
		
		Построим асимптотические доверительные интервалы в модели $ X_1, \ldots, X_n \sim \mathcal{N}(0, \theta) $
		на основе оценки максимального правдоподобия 
		$ T = \overline{X^2} $ (см. домашнюю работу 3).
		Эта оценка состоятельна для $ \theta $ и имеет асимптотическую дисперсию $ 2\theta^2 $ (см. домашнюю работу 4).
		Тогда оценка $ \sqrt{2}T $ состоятельна для $ \sqrt{2}\theta $.
		$$ \prob_\theta(a \leqslant \tfrac{\sqrt{n}(T - \theta)}{\sqrt{2}\theta}
		\leqslant b) \underset{n \to +\infty}{\to} \Phi(b) - \Phi(a), $$
		$$ \prob_\theta(a \leqslant \tfrac{\sqrt{n}(T - \theta)}{\sqrt{2}T}
		\leqslant b) \underset{n \to +\infty}{\to} \Phi(b) - \Phi(a), $$
		$$ \prob_\theta(T(1 - \tfrac{b\sqrt{2}}{\sqrt{n}}) \leqslant \theta
		\leqslant T(1 - \tfrac{a\sqrt{2}}{\sqrt{n}})) \underset{n \to +\infty}{\to} \Phi(b) - \Phi(a). $$
		
		Для $ \Phi(b) - \Phi(a) = 1 - \alpha $ положим $ a = z_{\frac{\alpha}{2}}, b = z_{1 - \frac{\alpha}{2}} $.
		Тогда доверительным будет интервал $ (T(1 - \tfrac{z_{1 - \frac{\alpha}{2}}\sqrt{2}}{\sqrt{n}}), T(1 - \tfrac{z_{\frac{\alpha}{2}}\sqrt{2}}{\sqrt{n}})) $.
		
	\end{problem}
	
	\begin{problem}[Задача 2]{inprocessing}
		
		Построим асимптотические доверительные интервалы в модели $ X_1, \ldots, X_n \sim \mathrm{Cauchy}(\theta) $
		на основе оценки максимального правдоподобия 
		$ \hat{\theta}_n $.
		Эта оценка состоятельна для $ \theta $ и имеет асимптотическую дисперсию $ 2 $ (см. домашнюю работу 4).
		Тогда
		$$ \prob_\theta(a \leqslant \tfrac{\sqrt{n}(\hat{\theta}_n - \theta)}{\sqrt{2}}
		\leqslant b) \underset{n \to +\infty}{\to} \Phi(b) - \Phi(a), $$
		$$ \prob_\theta(\hat{\theta}_n - \tfrac{b\sqrt{2}}{\sqrt{n}} \leqslant \theta
		\leqslant \hat{\theta}_n - \tfrac{a\sqrt{2}}{\sqrt{n}}) \underset{n \to +\infty}{\to} \Phi(b) - \Phi(a). $$
		
		Для $ \Phi(b) - \Phi(a) = 1 - \alpha $ положим $ a = z_{\frac{\alpha}{2}}, b = z_{1 - \frac{\alpha}{2}} $.
		Тогда доверительным будет интервал 
		$ (\hat{\theta}_n - \tfrac{z_{1 - \frac{\alpha}{2}}\sqrt{2}}{\sqrt{n}}, \hat{\theta}_n - \tfrac{z_{\frac{\alpha}{2}}\sqrt{2}}{\sqrt{n}}) $.
	
	\end{problem}
	
	\begin{problem}[Задача 3 (равномерное распределение)]{inprocessing}
		
		Пусть $ X_1, \ldots, X_n \sim \mathrm{R}[0, \theta] $.
		Пусть $ Y_i = \tfrac{X_i}{\theta} \sim \mathrm{R}[0, 1] $.
		Оценка максимального правдоподобия в модели есть $ X_{(n)} $ (см. домашнюю работу 3).
		Тогда случайные величины $ \tfrac{X_{(n)}}{\theta} $ и $ Y_{(n)} $ имеют одинаковые распределения.
		Имеем
		$ F_{\frac{X_{(n)}}{\theta}, \theta}(x) = F_{Y_{(n)}}(x) = x^nI(0 < x < 1) + I(x > 1) $.
		Тогда из равенства $ F_{\frac{X_{(n)}}{\theta}, \theta}(q_{\alpha}) = \alpha $
		получаем $ q_\alpha = \sqrt[n]{\alpha} $.
		
		Пусть $ \prob_\theta(q_1 < \tfrac{X_{(n)}}{\theta} < q_2) = 1 - \alpha $.
		Тогда $ \prob_\theta(\tfrac{X_{(n)}}{q_2} < \theta < \tfrac{X_{(n)}}{q_1}) = 1 - \alpha $
		и $ q_2^n - q_1^n = 1 - \alpha $.
		Функция $ l(q_2) = \tfrac{1}{\sqrt[n]{q_2^n - (1 - \alpha)}} - \tfrac{1}{q_2} $ имеет производную
		$$ \tfrac{1}{q_2^2} - \tfrac{q_2^{n - 1}}{(q_2^n - (1 - \alpha))^{
		\frac{n + 1}{n}}} = \tfrac{q_2^{n + 1} - (q_2^n - (1 - \alpha))^{
		\frac{n + 1}{n}}}{q_2^2(q_2^n - (1 - \alpha))^{
		\frac{n + 1}{n}}}. $$
		Поскольку $ 1 - \alpha > 0 $, то производная всегда принимает положительные значения
		и, следовательно, функция $ l $ принимает наименьшее значение при наибольшем допустимом значении $ q_2 $.
		Поскольку $ 0 \leqslant q_2 \leqslant 1 $, то минимум достигается при $ q_2 = 1 $.
		Тогда $ q_1 = \sqrt[n]{\alpha} $.
		Следовательно, точный доверительный интервал имеет вид $ (X_{(n)}, \tfrac{X_{(n)}}{\sqrt[n]{\alpha}}) $.
		
			
	\end{problem}
	
	\begin{problem}[Задача 3 ($ f_\theta(x) = e^{-(x - \theta)}I(x > \theta) $)]{inprocessing}
		
		Пусть случайные величины $ X_1, \ldots, X_n $ распределены
		с плотностью $ f_\theta(x) = e^{-(x - \theta)}I(x > \theta) $.
		Пусть $ Y_i = X_i - \theta \sim \mathrm{Exp}(1) $.
		Оценка максимального правдоподобия в модели есть $ X_{(1)} $ (см. домашнюю работу 3).
		Тогда случайные величины $ X_{(1)} - \theta $ и $ Y_{(1)} \sim \mathrm{Exp}(n) $ имеют одинаковые распределения.
		
		Пусть $ \prob_\theta(q_1 < X_{(1)} - \theta < q_2) = 1 - \alpha $.
		Тогда $ \prob_\theta(X_{(1)} - q_2 < \theta < X_{(1)} - q_1) = 1 - \alpha $
		и $ (1 - e^{-nq_2}) - (1 - e^{-nq_1}) = 1 - \alpha. $
		Отсюда $ 1 - \alpha = e^{-nq_1} - e^{-nq_2} = e^{-nq_1}(1 - e^{-n(q_2 - q_1)}) $.
		Тогда наименьшее значение разности $ q_2 - q_1 $ достигается при наименьшем возможном $ q_1 $,
		то есть при $ q_1 = 0 $. В этом случае $ q_2 = -\tfrac{1}{n}\ln \alpha $ (поскольку $ \alpha < 0 $, то $ q_2 $ будет положительным).
	
		Точный доверительный интервал будет иметь вид 
		$ (X_{(1)} + \tfrac{1}{n}\ln \alpha, X_{(1)}) $.
		
	\end{problem}
	
	\begin{problem}[Задача 4 ($ \mathcal{N}(\theta, \theta) $)]{inprocessing}
		
		Построим точный доверительный интервал для $ \theta > 0 $ в случае 
		$ X_1, \ldots, X_n \sim \mathcal{N}(\theta, \theta) $ на основе $ \overline{X} $.
		Имеем $ \tfrac{X_i - \theta}{\sqrt{\theta}} \sim \mathcal{N}(0, 1) $
		и $ \tfrac{(\overline{X} - \theta)\sqrt{n}}{\sqrt{\theta}} \sim \mathcal{N}(0, 1) $.
		
		Имеем $$ \prob_{\theta}(-z_{1 - \frac{\alpha}{2}} < \tfrac{(\overline{X} - \theta)\sqrt{n}}{\sqrt{\theta}}
		< z_{1 - \frac{\alpha}{2}}) = 1 - \alpha. $$
		Отсюда
		$$ \prob_{\theta}(-\tfrac{\sqrt{\theta}}{\sqrt{n}}z_{1 - \frac{\alpha}{2}} < \overline{X} - \theta
		< \tfrac{\sqrt{\theta}}{\sqrt{n}}z_{1 - \frac{\alpha}{2}}) = 1 - \alpha, $$
		$$ \prob_{\theta}((\overline{X} - \theta)^2
		< \tfrac{\theta}{n}z_{1 - \frac{\alpha}{2}}^2) = 1 - \alpha, $$
		Корни уравнения $ (\overline{X} - \theta)^2 = \tfrac{\theta}{n}z_{1 - \frac{\alpha}{2}}^2 $
		относительно $ \theta $ равны 
		$$ \overline{X} + \tfrac{z_{1 - \frac{\alpha}{2}}^2}{2n} \pm 
		\tfrac{1}{2}\sqrt{\left(2\overline{X} + \tfrac{z_{1 - \frac{\alpha}{2}}^2}{n}\right)^2 - 4\overline{X}^2}. $$
		Тогда точный доверительный интервал есть
		$$ \left(
		\overline{X} + \tfrac{z_{1 - \frac{\alpha}{2}}^2}{2n} - 
		\tfrac{1}{2}\sqrt{\left(2\overline{X} + \tfrac{z_{1 - \frac{\alpha}{2}}^2}{n}\right)^2 - 4\overline{X}^2},
		\overline{X} + \tfrac{z_{1 - \frac{\alpha}{2}}^2}{2n} + 
		\tfrac{1}{2}\sqrt{\left(2\overline{X} + \tfrac{z_{1 - \frac{\alpha}{2}}^2}{n}\right)^2 - 4\overline{X}^2}
		\right). $$
		
	\end{problem}
	
	\begin{problem}[Задача 4 ($ \mathcal{N}(\theta, \theta^2) $)]{inprocessing}
		
		Построим точный доверительный интервал для $ \theta $ в случае 
		$ X_1, \ldots, X_n \sim \mathcal{N}(\theta, \theta^2) $ на основе $ \overline{X} $.
		Имеем $ \tfrac{X_i - \theta}{\theta} \sim \mathcal{N}(0, 1) $
		и $ \tfrac{(\overline{X} - \theta)\sqrt{n}}{\theta} \sim \mathcal{N}(0, 1) $.
		
		Имеем (для $ \theta > 0 $) $$ \prob_{\theta}(-z_{1 - \frac{\alpha}{2}} < \tfrac{(\overline{X} - \theta)\sqrt{n}}{\theta}
		< z_{1 - \frac{\alpha}{2}}) = 1 - \alpha. $$
		Отсюда
		$$ \prob_{\theta}(-\tfrac{\theta}{\sqrt{n}}z_{1 - \frac{\alpha}{2}} < \overline{X} - \theta
		< \tfrac{\theta}{\sqrt{n}}z_{1 - \frac{\alpha}{2}}) = 1 - \alpha, $$
		$$ \prob_{\theta}\left(\overline{X}\left(1 - \tfrac{z_{1 - \frac{\alpha}{2}}^2}{n}\right)^{-1} < \theta
		< \overline{X}\left(1 + \tfrac{z_{1 - \frac{\alpha}{2}}^2}{n}\right)^{-1} \right) = 1 - \alpha, $$
		Тогда точный доверительный интервал есть
		$$ \left(
		\overline{X}\left(1 - \tfrac{z_{1 - \frac{\alpha}{2}}^2}{n}\right)^{-1},
		\overline{X}\left(1 + \tfrac{z_{1 - \frac{\alpha}{2}}^2}{n}\right)^{-1}
		\right). $$
		
	\end{problem}
	
	
\end{document}
