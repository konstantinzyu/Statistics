% \documentclass[12pt]{article}
\documentclass[12pt]{amsart}

\pagestyle{plain}
\usepackage[margin=2cm]{geometry} 

\usepackage{amsmath,amssymb,amsfonts,enumerate,latexsym,amsthm,textcomp,wasysym}
\usepackage{hyperref}
\usepackage{subfiles}
%\usepackage{tocloft}

%\usepackage{indentfirst}
\usepackage{cancel}
\usepackage{graphicx}
% \graphicspath{{pictures/}}
% \DeclareGraphicsExtensions{.pdf,.png,.jpg}
%\usepackage{russian}

% Colors
\usepackage[dvipsnames]{xcolor}
\definecolor{linkcolor}{HTML}{0000FF} % цвет ссылок
\definecolor{urlcolor}{HTML}{0000FF} % цвет гиперссылок
\definecolor{citecolor}{HTML}{0000FF} % цвет ссылки на статью
\hypersetup{pdfstartview=FitH, linkcolor=linkcolor, urlcolor=urlcolor, citecolor=citecolor, colorlinks=true}
% Пробелы, отступы и выделения
\definecolor{todocolor}{HTML}{FF4500} % цвет todo
\definecolor{defcolor}{HTML}{EE5D0F} % цвет определений
\newcommand{\TODO}[1]{\textcolor{todocolor}{НУЖНО #1}}
\renewcommand\labelenumi{\rm (\arabic{enumi})}
\renewcommand\theenumi{\rm (\arabic{enumi})}
\definecolor{completed}{HTML}{32CD32}
\definecolor{inprocessing}{HTML}{D19A0F}

% Pictures and diagrams
\usepackage[matrix, arrow, curve]{xy} 
\usepackage{tikz-cd}
\usepackage{tikz}
\usetikzlibrary{shapes.geometric}
\usepackage{makecell}

\tikzset{
	symbol/.style={
		draw=none,
		every to/.append style={
			edge node={node [sloped, allow upside down, auto=false]{$#1$}}}
	}
}

\usepackage[utf8]{inputenc}
\usepackage[russian]{babel}
\usepackage{verbatim}
\makeatletter
\def\@settitle{\begin{center}%
		\baselineskip14\p@\relax
		\bfseries
		\large \@title
	\end{center}%
}
\makeatother


%%%%%%%%%%%%%%%%%%%%%%%%%%%%%%%%%%%%%%%%%%%%%%%%%%%%%%%%%%%%
% % commands for making comments
%\usepackage[dvipsnames]{xcolor}
\newcommand{\YP}[1]{\footnote{\textcolor{red}{YP: #1}}}
\newcommand{\yp}[1]{\leavevmode{\color{red}{#1}}}
% {\textcolor{orange}{#1}} 
\usepackage[normalem]{ulem}
%%%%%%%%%%%%%%%%%%%%%%%%%%%%%%%%%%%%%%%%%%%%%%%%%%%%%%%%%%%%

% \textheight=270mm
% \textwidth=190mm
% \voffset=-40mm
% \hoffset=-35mm
% \pagestyle{empty}
% 
% \\SLoppy

\emergencystretch=5pt

\setcounter{tocdepth}{4}
\setcounter{secnumdepth}{4}

% Theorems

\newtheorem{theorem}{Теорема}
\newtheorem*{definition}{Определение}
\newtheorem{proposition}[theorem]{Предложение}
\newtheorem{lemma}[theorem]{Лемма}
\newtheorem{corollary}[theorem]{Следствие}
\newtheorem*{remark*}{Замечание}


\theoremstyle{definition}

% Environments

\newenvironment{problem}[2][Problem name]{\indent \textcolor{#2}{\textbf{#1}} \indent}{\indent}
\newenvironment{squarestatement}[1][Statement]{\indent \textbf{[#1]} \indent}
{$ \hfill \lhd $ \indent}

% New Commands

% Set definition
\newcommand{\defineset}[2]{\left\{
	\left.
	#1 \
	\right\vert
	#2
	\right\}}

\newcommand{\Alt}{\mathfrak{A}}
\newcommand{\Sym}{\mathfrak{S}}
\newcommand{\D}{\mathrm{D}}
\newcommand{\Q}{\mathrm{Q}}
\newcommand{\rC}{\mathrm{C}}
\newcommand{\T}{\mathrm{T}}
\newcommand{\rO}{\mathrm{O}}
\newcommand{\I}{\mathrm{I}}
\newcommand{\CC}{\mathbb{C}}
\newcommand{\RR}{\mathbb{R}}
\newcommand{\FF}{\mathbb{F}}
\newcommand{\EE}{\mathbb{E}}

\newcommand{\calA}{\mathcal{A}}
\newcommand{\calB}{\mathcal{B}}
\newcommand{\calE}{\mathcal{E}}
\newcommand{\calP}{\mathcal{P}}

\newcommand{\GL}{\operatorname{GL}}
\newcommand{\SL}{\operatorname{SL}}
\newcommand{\PGL}{\operatorname{PGL}}
\newcommand{\PSL}{\operatorname{PSL}}
\newcommand{\SU}{\operatorname{SU}}
\newcommand{\SO}{\operatorname{SO}}
\newcommand{\diag}{\operatorname{diag}}
\newcommand{\characteristic}{\operatorname{char}}
\newcommand{\kk}{\Bbbk}
\newcommand{\Gal}{\mathrm{Gal}}

\newcommand{\Hom}{\mathrm{Hom}}
\newcommand{\projective}[1]{\mathrm{P^1}(#1)}
\newcommand{\Id}{\mathrm{Id}}
\newcommand{\Image}{\mathrm{Im} \,}
\newcommand{\Aut}[1]{\mathrm{Aut}\left(#1\right)}
\newcommand{\pr}[1]{\mathrm{pr}_{#1}}
\newcommand{\Rad}[1]{\mathrm{Rad}\left(#1\right)}
\newcommand{\Ann}[1]{\mathrm{Ann}\left(#1\right)}
\newcommand{\op}[1]{#1^{\mathrm{op}}}
\newcommand{\End}[2]{\mathrm{End}_{#2}\left(#1\right)}
\newcommand{\Ab}{\mathrm{Ab}}

\newcommand{\pp}{\mathfrak{p}}
\newcommand{\qq}{\mathfrak{q}}



% Кусочное определение функции
\newcommand{\definefuntwo}[4]{
	\begin{cases}
		#1, & #2; \\
		#3, & #4.
	\end{cases}
}

\newcommand{\definefunthree}[6]{
	\begin{cases}
		#1, & #2; \\
		#3, & #4; \\
		#5, & #6.
	\end{cases}
}

\newcommand{\definefunfour}[8]{
	\begin{cases}
		#1, & #2; \\
		#3, & #4; \\
		#5, & #6; \\
		#7, & #8.
	\end{cases}
}

\newcommand{\prob}{\operatorname{P}}
\newcommand{\events}{\mathfrak{F}}
\newcommand{\expect}{\operatorname{E}}
\newcommand{\disp}{\operatorname{D}}
\newcommand{\cov}{\operatorname{Cov}}

\newcommand{\params}{\Theta}

\newcommand{\red}[1]{{\color{red} #1}}
\newcommand{\blue}[1]{{\color{blue} #1}}

\title{Решения задач по математической статистике}
\author{Константин Зюбин}

%\pagenumbering{arabic}

\begin{document}
	
	\maketitle
	
%	\tableofcontents
	
%\section{Дз 2}

\begin{problem}[Задача 1а]{inprocessing}
	
	Пусть случайные величины $ X, Y, Z $ распределены по закону $ \mathrm{E}(\tfrac{1}{\theta}) $.
	
	Предварительно заметим, что в силу независимости и одинаковой распределённости выполнены равенства
	$$ \expect_\theta X = \expect_\theta Y = \expect_\theta Z = \theta, $$
	$$ \expect_\theta X^2 = \expect_\theta Y^2 = \expect_\theta Z^2 = 2\theta^2, $$
	$$ \expect_\theta XY = \expect_\theta YZ = \expect_\theta ZX = \theta^2. $$
	
	Вычислим условным математические ожидания.
	$$ a(X, Y) = \expect_\theta(XY + XZ - Y^2 \mid X, Y)
	= \expect_\theta(XY\mid X, Y) + \expect_\theta(XZ\mid X, Y) - \expect_\theta(Y^2 \mid X, Y) = $$  
	$$ = \red{XY + X\expect_\theta(Z) - Y^2} 
	= \blue{XY - \theta X - Y^2}. $$
	Далее,
	$$ b(Y, Z) = \expect_\theta(XY + XZ - Y^2 \mid Y, Z) 
	= \expect_\theta(X(Y + Z)\mid Y, Z) - Y^2
	= \red{(Y + Z)\expect_\theta (X) - Y^2} 
	= \blue{\theta(Y + Z) - Y^2}, $$
	$$ c(Z, X) = \expect_\theta(XY + XZ - Y^2 \mid Z, X) = \expect_\theta(XY\mid Z, X) + XZ - \expect_\theta(Y^2) = $$ 
	$$ = \red{X\expect_\theta (Y) + XZ - \expect_\theta(Y^2)}
	= \blue{\theta X + XZ - 2\theta^2}. $$
	
\end{problem}

\begin{problem}[Задача 1б]{inprocessing}
	
	Теперь вычислим условные математические ожидания величин $ a, b, c $:
	$$ a_X(X) = \expect_\theta(a(X, Y) \mid X) 
	= \expect_\theta\left(XY - Y^2 + X\expect_\theta (Z) \mid X\right) = $$ 
	$$ = \red{X\expect_\theta(Y) - \expect_\theta(Y^2) +  X\expect_\theta (Z)}
	= \theta X - 2\theta^2 +  \theta X
	= \blue{2\theta X - 2\theta^2}. $$
	
	$$ a_Y(Y) = \expect_\theta(a(X, Y) \mid Y) = \expect_\theta\left(XY - Y^2 + X\expect_\theta (Z) \mid Y
	\right) = $$ 
	$$ = \red{Y\expect_\theta(X) - Y^2 + \expect_\theta(X)\expect_\theta(Z)}
	= \theta Y - Y^2 +  \theta^2
	= \blue{\theta Y - Y^2 + \theta^2}. $$
		
	$$ b_Y(Y) = \expect_\theta(b(Y, Z) \mid Y) = \expect_\theta\left((Y + Z)\expect_\theta (X) - Y^2 \mid Y\right) = $$ 
	$$ = \red{\expect_\theta (X)(Y + \expect_\theta(Z)) - Y^2}
	= \blue{\theta Y - Y^2 + \theta^2}. $$
	
	$$ b_Z(Z) = \expect_\theta(b(Y, Z) \mid Z) = \expect_\theta\left((Y + Z)\expect_\theta (X) - Y^2 \mid Z\right) = $$ 
	$$ = \red{ \expect_\theta (X)(\expect_\theta(Y) + Z) - \expect_\theta(Y^2)}
	= \blue{ \theta Z - \theta^2 }. $$
	
	$$ c_Z(Z) = \expect_\theta(c(Z, X) \mid Z) 
	= \expect_\theta\left(X\expect_\theta (Y) + XZ - \expect_\theta(Y^2) \mid Z\right) = $$ 
	$$ = \red{ \expect_\theta (X)\expect_\theta (Y) + \expect_\theta (X) \cdot Z - \expect_\theta(Y^2)}
	= \theta Z + \theta^2 - 2\theta^2
	= \blue{\theta Z - \theta^2}. $$
	
	$$ c_X(X) = \expect_\theta(c(Z, X) \mid X) 
	= \expect_\theta\left(X\expect_\theta (Y) + XZ - \expect_\theta(Y^2) \mid X\right) = $$ 
	$$ = \red{ X\expect_\theta (Y) + X \cdot \expect_\theta (Z) - \expect_\theta(Y^2)}
	= \blue{ 2\theta X - 2\theta^2}. $$
	
\end{problem}

\begin{problem}[Задача 1в]{inprocessing}
	
	Вычислим матожидания $ XY + XZ - Y^2, a, b, c, a_X, a_Y, b_Y, b_Z, c_Z, c_X $ и убедимся, что они совпадают.
	
	$$ \expect_\theta(XY + XZ - Y^2) =
	 \expect_\theta(X)\expect_\theta(Y) + \expect_\theta(X)\expect_\theta(Z) - \expect_\theta(Y^2) = $$ 
	 $$ = \theta^2 + \theta^2 - 2\theta^2 = 0. $$
	 
	 $$ \expect_\theta(a(X, Y)) 
	 = \expect_\theta \left(XY - Y^2 + X\expect_\theta (Z) \right)
	 = \expect_\theta(X)\expect_\theta(Y) - \expect_\theta(Y^2)  
	 + \expect_\theta (X)\expect_\theta (Z) = $$ 
	 $$ = \theta^2 - 2\theta^2 + \theta^2 = 0. $$
	 
	 $$ \expect_\theta(b(Y, Z)) 
	 = \expect_\theta \left((Y + Z)\expect_\theta (X) - Y^2 \right)
	 = \expect_\theta(Y)\expect_\theta(X) + \expect_\theta(Z)\expect_\theta(X) - \expect_\theta(Y^2) = $$ 
	 $$ = \theta^2 + \theta^2 - 2\theta^2 = 0. $$
	
	 $$ \expect_\theta(c(Z, X)) 
	 = \expect_\theta \left(X\expect_\theta (Y) + XZ - \expect_\theta(Y^2) \right)
	 = \expect_\theta(X)\expect_\theta(Y) + \expect_\theta(X)\expect_\theta(Z) - \expect_\theta(Y^2) = $$ 
	 $$ = \theta^2 + \theta^2 - 2\theta^2 = 0. $$
	 
	 $$ \expect_\theta(a_X(X)) = \expect_\theta(2\theta X - 2\theta^2) 
	 = 2\theta^2 - 2\theta^2 = 0. $$
	 $$ \expect_\theta(a_Y(X)) = \expect_\theta(\theta Y - Y^2 + \theta^2) 
	 = \theta^2 - 2\theta^2 + \theta^2 =  0. $$
	 $$ \expect_\theta(b_Y(X)) = \expect_\theta(\theta Y - Y^2 + \theta^2) 
	 = \theta^2 - 2\theta^2 + \theta^2 = 0. $$
	 $$ \expect_\theta(b_Z(X)) = \expect_\theta(\theta Z - \theta^2 )
	 = \theta^2 - \theta^2 =  0. $$
	 $$ \expect_\theta(c_Z(X)) = \expect_\theta(\theta Z - \theta^2) 
	 = \theta^2 - \theta^2 = 0. $$
	 $$ \expect_\theta(c_X(X)) = \expect_\theta(2\theta X - 2\theta^2) 
	 = 2\theta^2 - 2\theta^2 = 0. $$
	 
\end{problem}

\begin{problem}[Задача 2а (распределение Пуассона)]{inprocessing}
	
	Пусть имеются независимые случайные величины $ X_1, \ldots, X_n \sim \mathrm{Poiss}(\theta) $.
	Вычислим условные математические ожидания.
	$$ \expect_\theta(X_1 \mid X_1 + \ldots + X_n)
	= \tfrac{1}{n}\sum\limits_{i = 1}^{n}\expect_\theta(X_i \mid X_1 + \ldots + X_n)
	= \tfrac{1}{n}\expect_\theta(X_1 + \ldots + X_n \mid X_1 + \ldots + X_n)
	= \tfrac{\sum\limits_{i = 1}^{n} X_i}{n} = \overline{X}. $$
	
	Вычислим $$ \prob_\theta(X_1 = k \mid X_1 + \ldots + X_n = m)
	= \tfrac{\prob_\theta(X_1 = k, X_2 + \ldots + X_n = m - k)}{\prob_\theta(X_1 + \ldots + X_n = m)}
	= \tfrac{e^{-\theta}\theta^k}{k!} \cdot \tfrac{e^{-(n - 1)\theta}((n - 1)\theta)^{m - k}}{(m - k)!}
	\cdot \tfrac{m!}{e^{-n\theta}(n\theta)^m} = $$
	$$ = \tfrac{m!}{k!(m - k)!} \cdot \left(\tfrac{1}{n}\right)^{k} \cdot \left(\tfrac{n - 1}{n}\right)^{m - k}. $$
	Отсюда
	$$ \sum\limits_{k = 0}^{m} k^2 \cdot \tfrac{m!}{k!(m - k)!} \cdot \left(\tfrac{1}{n}\right)^{k} \cdot \left(\tfrac{n - 1}{n}\right)^{m - k}
	= \tfrac{m}{n} \sum\limits_{k = 1}^{m} k \cdot \tfrac{(m - 1)!}{(k - 1)!(m - k)!} \cdot \left(\tfrac{1}{n}\right)^{k - 1} \cdot \left(\tfrac{n - 1}{n}\right)^{m - k} = $$ 
	$$ = \tfrac{m}{n}\left.\expect_\theta(X_1 + 1 \mid X_1 + \ldots + X_n)\right|_{X_1 + \ldots + X_n = m - 1}
	= \tfrac{m}{n}\left(\tfrac{m - 1}{n} + 1\right) = \tfrac{m}{n} + \tfrac{m(m - 1)}{n^2}. $$
	Поэтому
	$$ \expect_\theta(X_1^2 \mid X_1 + \ldots + X_n) = \overline{X} + \overline{X}(\overline{X} - \tfrac{1}{n}). $$
	
\end{problem}
	
\begin{problem}[Задача 2б (геометрическое распределение)]{inprocessing}	
	
	Пусть имеются независимые случайные величины $ X_1, \ldots, X_n \sim \mathrm{Geom}(\theta) $.
	Вычислим условные математические ожидания.
	$$ \expect_\theta(X_1 \mid X_1 + \ldots + X_n)
	= \tfrac{1}{n}\sum\limits_{i = 1}^{n}\expect_\theta(X_i \mid X_1 + \ldots + X_n)
	= \tfrac{1}{n}\expect_\theta(X_1 + \ldots + X_n \mid X_1 + \ldots + X_n)
	= \tfrac{\sum\limits_{i = 1}^{n} X_i}{n} = \overline{X}. $$
	
	Найдём распределение свёртки геометрических случайных величин:
	$$ \prob(X_1 + \ldots + X_n = m)
	= \sum\limits_{x_1 + \ldots + x_n = m} \prod\limits_{i = 1}^{n} \prob(X_i = x_i)
	= \sum\limits_{x_1 + \ldots + x_n = m} \prod\limits_{i = 1}^{n} \theta(1 - \theta)^{x_i - 1}
	= \tfrac{(m - 1)!}{(m - n)!(n - 1)!} \theta^n(1 - \theta)^{m - n}. $$
	
	Вычислим (для $ m \geqslant k + n - 1 $) условную вероятность
	$$ \prob_\theta(X_1 = k \mid X_1 + \ldots + X_n = m)
	= \tfrac{\prob_\theta(X_1 = k, X_2 + \ldots + X_n = m - k)}{\prob_\theta(X_1 + \ldots + X_n = m)} = $$  
	$$ = \theta(1 - \theta)^{k - 1} \cdot
	\tfrac{(m - k - 1)!}{(m - k - n + 2)!(n - 2)!} \theta^{n - 1}(1 - \theta)^{m - k - (n - 1)}
	\cdot
	\tfrac{(m - n)!(n - 1)!}{(m - 1)!} \tfrac{1}{\theta^n(1 - \theta)^{m - n}} = $$
	$$ = (C_{m - 1}^{n - 1})^{-1} \cdot C_{m - k - 1}^{n - 2}. $$
	Отсюда
	$$ (C_{m - 1}^{n - 1})^{-1} \cdot \sum\limits_{k = 1}^{m - n + 1} k^2 C_{m - k - 1}^{n - 2}
	= (C_{m - 1}^{n - 1})^{-1} \cdot \sum\limits_{j = n - 2}^{m - 2} (m - 1 - j)^2C_{j}^{n - 2}. $$ 
	Так как
	$$ \sum\limits_{j = n - 2}^{m - 2} C_{j}^{n - 2}
	= 1 + C_{n - 1}^{n - 2} + \ldots + C_{m - 2}^{n - 2}
	= C_{m - 1}^{n - 1} $$
	и
	$$ \sum\limits_{j = n - 2}^{m - 2} (j + 1)C_{j}^{n - 2}
	= (n - 1)\sum\limits_{j = n - 2}^{m - 2} C_{j + 1}^{n - 1}
	= (n - 1)C_{m}^{n}, $$
	а также
	$$ \sum\limits_{j = n - 2}^{m - 2} (j + 1)^2C_{j}^{n - 2}
	= \sum\limits_{j = n - 2}^{m - 2} ((j + 2)(j + 1) - (j + 1))C_{j}^{n - 2}
	= n(n - 1)C_{m + 1}^{n + 1} - (n - 1)C_{m}^{n}, $$
	то
	$$ (C_{m - 1}^{n - 1})^{-1} \cdot \sum\limits_{k = 1}^{m - n + 1} k^2 C_{m - k - 1}^{n - 2}
	= (C_{m - 1}^{n - 1})^{-1} \cdot \left(m^2C_{m - 1}^{n - 1} - 2m(n - 1)C_{m}^{n} + n(n - 1)C_{m + 1}^{n + 1} - (n - 1)C_{m}^{n}\right) = $$
	$$ =
	m^2 - \tfrac{2m(n - 1)m}{n} + \tfrac{n(n - 1)m(m + 1)}{n(n + 1)} - \tfrac{m(n - 1)}{n} = $$  $$
	= \tfrac{m}{n(n + 1)}\left(mn(n + 1) -2m(n - 1)(n + 1) + n(n - 1)(m + 1) - (n - 1)(n + 1)\right) = $$  $$
	= \tfrac{m}{n(n + 1)} \cdot \left(m(n^2 + n - 2n^2 + 2 + n^2 - n) + n^2 - n	-n^2 + 1 \right)
	=\tfrac{m(2m - n + 1)}{n(n + 1)}.  $$
	
	Тогда условное матождание равно $ g(S_n) = \tfrac{S_n(2S_n -n + 1)}{n(n + 1)} $.
	
\end{problem}

\begin{problem}[Задача 3]{inprocessing}	

	Пусть случайная величина $ X $ имеет распределение $ \mathrm{E}(1) $
	и случайная величина $ Y \mid X $ распределена так же, как $ \exp(X) $.
	Вычислим совместную плотность для $ (X, Y) $ 
	и условные матожидания $ \expect(X^2 \mid Y), \expect (X \mid Y) $.
	
	Имеем
	$ f_{Y \mid X}(y \mid x) = xe^{-yx}I(y > 0) $ и $ f_X(x) = e^{-x}I(x > 0) $.
	Тогда совместная плотность равна $ f_{X, Y}(x, y) = xe^{-x(y + 1)}I(x > 0, y > 0). $
 	
 	Далее,
 	$$ f_{Y}(y) = \int\limits_{-\infty}^{+\infty} xe^{-x(y + 1)}I(x > 0, y > 0)dx
 	= I(y > 0) \cdot \int\limits_{0}^{+\infty} xe^{-x(y + 1)}dx = $$
 	$$ = I(y > 0) \cdot \left(\left.-\tfrac{xe^{-x(y + 1)}}{y + 1}\right|_{x=0}^{+\infty} 
 	+ \int\limits_{0}^{+\infty} \tfrac{e^{-x(y + 1)}}{y + 1}dx \right) = \tfrac{I(y > 0)}{(y + 1)^2}. $$
 	Тогда
 	$$ f_{X \mid Y}(x \mid y) = \tfrac{f_{X, Y}(x, y)}{f_{Y}(y)}
 	= x(y + 1)^2e^{-x(y + 1)}I(x > 0, y > 0). $$
 	Поэтому
 	$$ \expect(X \mid Y)(Y) = \int\limits_{-\infty}^{+\infty} x^2(Y + 1)^2e^{-x(Y + 1)}I(x > 0, Y > 0)dx = $$  $$
 	= (Y + 1)^2I(Y > 0)\int\limits_{0}^{+\infty} x^2e^{-x(Y + 1)}dx
 	= (Y + 1)^2I(Y > 0) \cdot \tfrac{2}{(Y + 1)^3} = \tfrac{2I(Y > 0)}{Y + 1}. $$
 	Вычислим второе условное матожидание:
 	$$ \expect(X^2 \mid Y)(Y) = \int\limits_{-\infty}^{+\infty} x^3(Y + 1)^2e^{-x(Y + 1)}I(x > 0, Y > 0)dx = $$  $$
 	= (Y + 1)^2I(Y > 0)\int\limits_{0}^{+\infty} x^3e^{-x(Y + 1)}dx
 	= (Y + 1)^2I(Y > 0) \cdot \tfrac{6}{(Y + 1)^4} = \tfrac{6I(Y > 0)}{(Y + 1)^2}. $$
 	
\end{problem}

\end{document}
